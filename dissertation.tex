
% example for dissertation.sty
\documentclass[
  % Replace oneside by twoside if you are printing your thesis on both sides
  % of the paper, leave as is for single sided prints or for viewing on screen.
  oneside,
  %twoside,
  11pt, a4paper,
  footinclude=true,
  headinclude=true,
  cleardoublepage=empty
]{scrbook}


\usepackage{dissertation}

% ACRONYMS -----------------------------------------------------

%import the necessary package with some options
\usepackage[acronym,nonumberlist,nomain]{glossaries}

%enable the following to avoid links from the acronym usage to the list
%\glsdisablehyper

%displays the first use of an acronym in italic
\defglsdisplayfirst[\acronymtype]{\emph{#1#4}}

%the style of the Glossary
%\setglossarystyle{listgroup}

%set the name for the acronym entries page
\renewcommand{\glossaryname}{Acronyms}

%this shall be the last thing in the acronym configuration!!
%
\makeglossaries

%load acronyms file
%\loadglsentries[\acronymtype]{sec/acronyms}

%% **MORE INFO** %%

%to add the acronyms list add the following where you want to print it:
%\printglossary[type=\acronymtype]
%\clearpage
%\thispagestyle{empty}

%to use an acronym:
%\gls{qps}

% compile the thesis in command line with the following command sequence:
% pdlatex dissertation.tex
% makeglossaries dissertation
% bibtex dissertation
% pdlatex dissertation.tex
% pdlatex dissertation.tex

% ----------------------------------------------------------------

% Title
\titleA{Development of a system}
\titleB{compliant with the Application-layer}
\titleC{Traffic Optimization protocol}
% Author
\author{Paulo Edgar Mendes Caldas}

% Supervisor(s)
\supervisor{Pedro Nuno Miranda de Sousa}
%\cosupervisor{}

% University (uncomment if you need to change default values)
% \def\school{Escola de Engenharia}
% \def\department{Departamento de Inform\'{a}tica}
% \def\university{Universidade do Minho}
\def\masterdegree{Informatics Engineering}

% Date
\date{\myear} % change to text if date is not today

% Keywords
%\keywords{master thesis}

% Glossaries & Acronyms
%\makeglossaries  %  either use this ...


%!TEX root = ../dissertation.tex

\newacronym{p2p}{P2P}{Peer-to-peer}
\newacronym{isp}{ISP}{Internet Service Provider}
\newacronym{as}{AS}{Autonomous System}
\newacronym{cdn}{CDN}{Content Distribution Network}
\newacronym{emea}{EMEA}{Europe, the Middle East and Africa}
\newacronym{qos}{QoS}{Quality of Service}
\newacronym{ietf}{IETF}{Internet Engineering Task Force}
\newacronym{dht}{DHT}{Distributed Hash Table}
\newacronym{can}{CAN}{Content Addressable Network}
\newacronym{qoe}{QoE}{Quality of Experience}
\newacronym{igp}{IGP}{Interior Gateway Protocol}
\newacronym{egp}{EGP}{Exterior Gateway Protocol}
\newacronym{mtr}{MTR}{Multi-Topology Routing}
\newacronym{mpls}{MPLS}{Multiprotocol Label Switching}
\newacronym{diffserv}{DiffServ}{Differentiated services}
\newacronym{cdni}{CDNI}{Content Distribution Network Interconnection}
\newacronym{sdn}{SDN}{Software Defined Networking}
\newacronym{rtt}{RTT}{Round-Trip Time}
\newacronym{alto}{ALTO}{Application-Layer Traffic Optimization}
\newacronym{padis}{PaDIS}{Provider-aided Distance Information System}
\newacronym{cate}{CaTE}{Content-aware Traffic Engineering}
\newacronym{netpaas}{NetPaaS}{Network Platform as a Service}
\newacronym{xmpp}{XMPP}{Extensible Messaging and Presence Protocol}
\newacronym{pid}{PID}{Provider-Defined Identifier}
\newacronym{sql}{SQL}{Structured Query Language}
\newacronym{dns}{DNS}{Domain Name System}
\newacronym{ip}{IP}{Internet Protocol}
\newacronym{http}{HTTP}{Hypertext Transfer Protocol}
\newacronym{dos}{DoS}{Denial of Service}
\newacronym{gslb}{GSLB}{Global Server Load Balancing}
\newacronym{url}{URL}{Uniform Resource Locator}
\newacronym{gnp}{GNP}{Global Network Positioning}
\newacronym{idmaps}{IDMaps}{Internet Distance Map Service}

\glsaddall[types={\acronymtype}]
\makeindex

\ummetadata

\begin{document}

\umfrontcover
\umtitlepage

\chapter{Acknowledgements}

    I would like to firstly thank my advisor, professor Pedro Nuno Sousa, who was always present in any moment I struggled and required input to improve on my work.

    I would also like to thank my family for financially and emotionally supporting me through my academic journey, the friends I've made along the way that made me see the best in people, and last but not least my dog Oscar who showed me unconditional love like only a dog could.

    I finally also thank you, the reader - as a work unused is no work at all, may you find some value in it.

\chapter*{Abstract}

    With the ever-increasing global Internet usage that is following the start of the new decade, the need to optimize this world-scale network of computers becomes a big priority in the technological sphere that has the number of users increasing, as are the Quality of Service (QoS) demands by applications in domains such as media streaming or virtual reality.

    In the face of rising traffic and stricter application demands, a better understanding of how Internet Service Providers (ISPs) should maintain their assets is needed.
    As an effort to optimize the Internet, one important concern is how applications utilize the underlying network infrastructure to achieve their goals.
    Peer-to-peer (P2P) applications, for example, have been classically unfavored by ISPs due to their unpredictable and uncontrollable nature, as well as a less than efficient utilization of network resources, resulting in higher costs and non-achievable standards of QoS for ISPs.
    An evident issue is that most of these applications act with little regards for ISP preferences, as can be evidenced by the lack of care by P2P applications in achieving network proximity among neighboring peers, a feature that would be preferable by network administrators and that could also improve application performance.
    Even a best effort attempt by applications to compromise with ISP policies is hard to be successful if such policies aren't clearly communicated to them, and as such a framework to bridge layer interests has much potential.

    This thesis aims to implement and extend upon the Application-Layer Traffic Optimization (ALTO) working group, which devised a request-response protocol where privileged and trustworthy entities provide guidance to applications in the form of network status information and administrative preferences, with the intent of achieving layer cooperation during normal application operations as a means to achieve better Internet harmony by maximizing optimal resource usage and minimizing costs.


\chapter{Resumo}

\todo{Finish this}
\textbf{[Translate abstract into portuguese]}


\clearpage
\tableofcontents
\listoffigures
\listoftables
\printglossary[type=\acronymtype]

\clearpage

\thispagestyle{empty}
\pagenumbering{arabic}

\chapter{Introduction}

\section{Context and motivation}

    As society as a whole advances, so does seem to increase the individual's quality of life, which in turn increases the standard to be expected from the society he lives in.
    As such, technology itself must quickly adapt to the needs of the people it serves, whichever they may be - educational, medical, logistical, just to name a few - and consistently create or improve upon solutions that inevitably change the day-to-day living of the many that use or reap the benefits of such solutions.
    A particular example that is still fresh in this generation is in the relationship between people and computers - where they may have been nonexistent a century ago, reserved for industries fifty years ago and valuable household commodity a few decades ago, it is now common to see a family home with more than a dozen computers, with a variety fitting for the many needs they can solve.
    The increased number of computers and their expected functionalities has made it so computer networking as a whole has to be improved upon.

    The internet allows computers to connect to one another in a worldwide network that applications can use to further increase their possibilities.
    However, when certain applications go unchecked it becomes very difficult for ISPs because such applications can create traffic which is either impossible, infeasible, or too costly to manage.
    This issue is further exacerbated when considering the scale of the next decade, where Cisco predicts that by 2022 global internet users will make up 60 percent of the world's population, and global IP traffic will reach 396 exabytes per month \cite{cisco2019}.
    The problem of network management will thus increase in difficulty due to the sheer scale of Internet usage, and to provide certain service standards to applications requires traffic engineering solutions.

    Considering a network of computers which are running applications to fit a given use case for the user, such as transferring a file, watching a real-time video, or consuming the content of a given social network, these applications are responsible for creating traffic that must be supported by the underlying network infrastructure, meaning all the hardware and software that is utilized by given companies to provide to end users the ability to communicate with each other.
    These applications can be thought of as citizens of a communications facility that provides the service of accessibility to other citizens, and there is a common incentive in maintaining this facility in such a way that keeps the service up to its standards.
    As such, and like any other community-shared facility, it must be maintained by the owners, and part of it includes creating and enforcing policies that uphold the facility's quality.
    During the runtime of an application that is connected to a network, the way it is programmed to operate has impact on the traffic it generates on the network, and thus how resourceful it is with the shared domain it uses.
    The logic of the program dictates how the shared network is used to achieve a given goal, and how it accomplishes it can be more or less preferable by the service providers - for example, which hosts to consume a service from, at what time of the day some traffic is generated, how much traffic is needed to achieve a use case, etc.
    Peer-to-peer (P2P) applications are an infamous example of a kind of application that often makes decisions that are not preferable by ISPs.
    These applications create overlay networks, which are abstract networks constructed on top of the underlying network that supports it, and on which the application's logic runs on, essentially making it infrastructure-agnostic.
    Historically, P2P traffic has not been preferable by ISPs due to its unpredictable and hard to manage nature.
    Indeed, if P2P applications simply keep an overlay connection between peers that does not span more than a few hops, whilst ignorant to them being, for example, either direct network links or spanning multiple Autonomous Systems (ASs) in the underlay, the generated traffic is always at risk of being inefficient and taxing on the supporting infrastructure if the application's logic does not make preferable decisions on how to use the shared domain - for example, by neighboring other peers which reside outside network borders, which are more expensive to reach.
    As file-sharing traffic currently uses around 7 exabytes per month (including P2P-based file-sharing) \cite{cisco2019}, and BitTorrent alone makes up 27\% percent of total upstream volume of traffic \cite{sandvine2019} it's in the best interest of both ISPs and P2P applications a way for the overlay and underlay levels to operate in synergy, i.e. how to combine efforts to guarantee that the needs of both layers are met.

    Current consumer trends suggest that media consumption will make up a considerable part of global internet traffic.
    In fact, Cisco predicts that, by 2022, more than 82 percent of all consumer internet traffic will be dedicated to Internet video streaming and downloads, and Content Distribution Networks (CDNs) will carry 72 percent of all internet traffic \cite{cisco2019} \todo{perguntar professor se whitepaper editado faz mal}.
    CDNs act by injecting content geographically nearby end users to increase availability and reduce total traffic usage, and are an example of how applications can better leverage the shared domain's resources to achieve their goal.
    The CDNs' management layer itself can optimize their application behaviour in ways that are advantageous to both applications using the CDNs and the shared network structure, and such ways could include whose edge server to cache data to, how to efficiently match end users to appropriate edge servers, or how to increase service reachability among other CDNs.
    Thus, much like P2P networks, content distribution networks could also greatly benefit from cooperative interactions with network providers.
    These optimizations should be made by the parties which have economical interest in guaranteeing good performance of the overall ecosystem, i.e. those acting on the over and underlay, and should seek to, from both application and network administration input, understand how to utilize the given network resources to achieve their application needs in a way that is cheap, effective and sustainable.

    More broadly, most kinds of applications that generate traffic on a network could benefit from input by entities which know how such network is structured and what political and administrative biases exist.
    Of course, a one-sided approach could also exist to optimize resourcefulness of the network structure - applications could use an independent internal logic that utilizes measurements and knowledge of the outside world to better aid their decisions, and likewise ISPs can attempt to throttle, block, or generally engineer traffic that doesn't conform to their guidelines.
    In fact, these one-sided approaches are precisely what happens currently, but this work aims to argue for a two-sided cooperative approach.

    In short, the issue that motivates this thesis is the lack of proper cooperation between the overlay and underlay levels in the task of optimizing traffic that originates from decisions that occur at the application level, e.g., peer selection for file retrieval in file-sharing P2P applications, software distribution mirror selection, CDN provider server or cache redirections, high traffic load scheduling, etc.
    This problem is not new to the Internet Engineering Task Force (IETF) who devised a working group to explore possible IETF standardization on traffic localization after test results concluded that P2P applications that select peers based on exclusive network information provided by ISPs could reduce network infrastructural and administrative costs as well as increase application download rates \cite{seedorf2009}.
    Such working group devised a request-response protocol by the same name, ALTO, where clients could query authoritative and trustworthy servers on information that regards to the underlay structure where the client operates.
    While P2P applications were the motivation for the ALTO working group to be created, the benefits of a standardized, maintained, and well provided system for network information querying and guidance on traffic-related decisions could help create the vision of ISPs and applications cooperating for mutual benefit, being thus advantageous for more than P2P applications - in essence, it would be a helpful system for any situation where a decision could be optimized with the addition of proper insight on network infrastructure.
    This work then focuses on tackling the theme of application-infrastructure cooperation on the Internet, with particular focus on the ALTO protocol as cooperation enabler.

\section{Objectives}

    The main objective of this thesis is to develop a working system that adheres and expands upon the ALTO working group's protocol and architecture.
    The starting point will be a preexisting software project that served as a proof of concept to the strategy of traffic optimization at the application layer, and which will now be extended in two ways: firstly, by restructuring and documenting the existing code in order to, through the compliance with the standards of object oriented programming and software development guidelines, present a solution that could be continuously maintained and modified; secondly, by further expanding on the software's functionality, e.g., adding more types of cost metrics, specifying meta-data which give the resources a time-specific applicability, specifying means of synchronizing data among servers, restricting user interaction via access control methods, etc).
    Whilst expanding upon the ALTO working group's devised solution is a goal, it is also important that the developed work complies with the specifications it is based on, so the work done by the IETF in regards to documentation and general reasoning of the protocol remains consistent with this implementation, with further additions being reasoned in this work.

    With the intent of completing its main goal, this work's partial objectives were devised as follows:

\begin{itemize}
    \item Literature review in regards to application-level traffic optimization and the cooperation (or lack thereof) between overlay networks and the underlay they operate on.
        More specifically, an understanding of the consensus on the existing issues, and an overview on currently proposed solutions.
    \item Complete overview of the ALTO working group's current work.
        More specifically, an overview of both their existing RFC documents and the currently active internet drafts being developed by them at the time of writing.
    \item Familiarization with the existing system to be worked on and definition of both a new system architecture which complies with and extends the ALTO solution, as well as the new function modules to be added and how they should operate.
    \item Implementation of both the devised solution as well as a bare-bones P2P file-sharing application for testing purposes.
    \item Construction of a realistic network simulation scenario where the P2P file-sharing application will operate in.
    \item Test of the implemented solution on the simulated scenario, and its analysis in comparison to preexisting strategies.
\end{itemize}

\section{Contributions}

\todo{this}

\section{Thesis organization}

    This dissertation will be organized in six chapters, as follows:

\begin{itemize}
    \item \textbf{Introduction}: Provides context to the problem to be attempted to solve, as well as motivation to attempt to do so. Coupled to this, the dissertation's main goal is presented.
    \item \textbf{State of the Art}: Display of the theory related to existing and popular technologies or overall concepts that could be targeted consumers of the ALTO protocol; Discussion of existing attempts to optimize application traffic using network information with and without close underlay cooperation; Overview of the ALTO working group's proposed protocol and architecture.
    \item \textbf{Specification}: Presentation of the devised system's functional and non-functional requirements, as well as an overview of the planned architecture.
    \item \textbf{Implementation}: Details to the decisions made and steps taken in the task of implementing the specified project.
    \item \textbf{Testing and result analysis}: Overviews the planned simulation scenario, how it was materialized, and how the tests were run. Additionally, provides the retrieved results from such simulations.
    \item \textbf{Conclusion}: Presents the results of this thesis in regards to what objectives were completed. Additionally, a critical analysis on the simulation results is made and argued against the initial hypothesis, arguments are made for the product's usefulness, and future work is suggested.
\end{itemize}{}


\chapter{State of the art}

    This chapter aims to provide a literature overview to the topics that relate to the main problem that this thesis aims to help solve, which is the lack of cooperation between applications and the infrastructure where they reside.
    As such, the first section focuses on discussing structural network patterns utilized by applications to give them particular properties which are helpful to achieve their use case.
    Among these patterns, particular interest will be given to three of them for the following reasons: firstly, due to their popularity in the current network paradigm; secondly, due to their potential to optimize traffic which is generated at the application level.
    These patterns are thus the classical server-client architecture, the distributed approach of peer-to-peer (P2P) architectures, and content distribution networks (CDNs).
    For each of these, a conceptual analysis is made - more specifically, contextual background, the architecture itself, advantages and disadvantages, and possible use cases.
    Additionally, there's an examination on how applications that utilize these networks affect, positively or negatively, the physical infrastructure where they operate on, and where does potential reside for mutually-beneficial cooperative behaviour between these two layers.
    The following section displays existing proposals for increased layer cooperation, and alongside it a discussion on the practical consequences of adopting such proposals.
    The final chapter gives special attention to the Application-Layer Traffic Optimization (ALTO) working group's proposal, as it is the baseline for this thesis's work..

\section{Peer-to-Peer (P2P) Networks}

\subsection{Concepts and Applications}

    Due to the many hybrid implementations that have surfaced, the definition of a P2P network has become harder to pinpoint.
    Nevertheless, a P2P network is grounded on some definitions, among them that it consists on many singular computing elements, the "peers", which have among themselves similar privileges and functions (this contrasts with the client-server architecture, where two different roles exist - the one that provides a service and the one that can consume it - with functionality and control being thus centralized).
    P2P networks decentralize computational resources as a means to achieve a given task in a way that is inherently different to a centralized counterpart.
    This decentralized architecture of the entire system as a whole gives it an interesting list of properties, among them:

\begin{itemize}
    \item \textbf{Dynamic scaling}: As all member nodes can share their computing resources with the network, the system increases its capacity with an increase in its users.
        Since the peers also act as clients to the network, scaling the service becomes less of a challenge as each new client will also act as a server.
        This also removes the necessity to manage how many service resources are needed - the amount of existing resources is linked to the number of existing clients, and thus there's no need to purchase and manage central resources, as the network dynamically allocates them by nature.
    \item \textbf{Resilience to failure}: Whereas centralized solutions are vulnerable to node and link failure, a decentralized one can more easily work around such threats - as all peers can encompass the same server functionality, network services and resources are not provided on a limited set on nodes, but instead can be redundantly deployed throughout as needed.
    \item \textbf{Power decentralization}: As a direct consequence of the sharing resources, no single peer has direct control of the network, and the information is not centralized.
        As such, this considerably deters any attempts to overpower the network, e.g., via means of censorship or biased node favouring.
\end{itemize}

    These, however, are not without their nuances - since many P2P hybrids exist, these properties are not immutable.
    For example, if we consider BitTorrent, which has Tracker servers to redirect users to a correct peer with the requested resource, whilst the network itself can still be resilient to failure, the content-retrieval service that the P2P network provides has a single point of failure and of control - the trackers themselves.
    Furthermore, the P2P network design brings, by its nature, alongside their potentially advantageous properties, also some some potentially disadvantageous ones to consider:

\begin{itemize}
    \item \textbf{Security hazards:} The equal functionality property that P2P networks have give peers much power influence others.
        Without proper care, malicious peers are a security risk.
    \item \textbf{Management:} Since resources and services are not centralized, tasks such as event logging and resource backups become very difficult, and perhaps impossible if the peers do not abide by any proper orchestration strategy.
\end{itemize}

    P2P applications have had, in the past decades, a mainstream image that is plagued with legality and security issues.
    Nonetheless, when overcome, the P2P networking strategy possesses many interesting properties - some of them displayed above - that make it fitting for varied use cases, e.g., file sharing, media streaming, social networking and problem solving via distributed and cooperative algorithms.

    Either way, the influence of P2P applications is undeniable: Sandvine's global Internet phenomena report concluded that BitTorrent alone had in 2019 a global application total traffic share of 2.4\%, and perhaps most importantly over 27\% of total upstream volume of traffic, and over 44\% in Europe, the Middle East and Africa (EMEA) alone \cite{sandvine2019}.
    Beyond file sharing purposes alone, P2P applications have been recently considered a fitting solution for low-cost content delivery systems in high demand scenarios - for example,  in applications such as PPStream in China which provide television content over IP to large audiences.
    Similarly, Akamai recognizes the potential of P2P technologies to provide a highly distributed option for serving static content over the network, although it being currently lacking in management and control features \cite{akamai-report}.
    Indeed, peer-to-peer Internet video broadcast services - and world-wide static content delivery services for that matter - seem attractive as they are cost-effective and easy to deploy, and are fitting to larger scale demands, and thus have the potential to become a more mainstream solution \cite{jianchuanliu2008}.

    Concluding, the P2P network architecture has many fitting use cases, and their rather different strategy, compared to the client-server architecture, to achieve its goals gives it many potentially interesting properties for users and ISPs.
    Considering its large global traffic share, particularly in upstream traffic, and its potential adoption towards the large scale demands of the future, P2P applications are likely to persist in the future and will be in the minds of ISP administrators.

\subsection{Architecture}

    As stated previously, the term "Peer-to-peer" has become very broad and now serves as an umbrella for many different variations of the core decentralized architecture.
    Thus, this chapter focuses not on overviewing a single conceptual architecture of what defines a P2P network, but instead of the many existing variations and how they differ among themselves.
    All P2P networks are characterized by consisting of peers that know one another as to form a so-called overlay network on top of its supporting network.
    How peers are organized in these P2P networks and how they operate is what distinguishes the many sub-types.
    Table \ref{table:p2p-structures} groups known P2P systems in regards to their centralization and structure, as did \cite{p2p-survey-1} and \cite{p2p-survey-2}, with the latter further distinguishing the protocols in regards to other metrics, e.g., security, reliability, and performance.
    The rest of this section follows the survey made by the former.


    One would expect that all P2P applications would have no centralization at all, since the P2P design ponders  spread throughout the network.
    However, some modifications have been made in some of these sub-types, which shift how much decentralization they have.
    Similarly, different strategies exist that dictate the structural hierarchy of the peers on the network.
    As would be expected, these sub-types of P2P networks thus possess different strengths and weaknesses, and these can be leveraged to the most appropriate use cases.

\begin{table}[]
\caption{Types of P2P systems (Adapted from \cite{p2p-survey-1})}
\label{table:p2p-structures}
\begin{adjustbox}{max width =1.1\textwidth,center}
\begin{tabular}{lllllll}
                                                 &                                        &                                                                                 &                                                                                                                 &                                                                                                                          &  &  \\ \cline{3-5}
                                                 & \multicolumn{1}{l|}{}                  & \multicolumn{3}{c|}{Centralization}                                                                                                                                                                                                                                                                                          &  &  \\ \cline{3-5}
                                                 & \multicolumn{1}{l|}{}                  & \multicolumn{1}{l|}{Hybrid}                                                     & \multicolumn{1}{l|}{Partial}                                                                                    & \multicolumn{1}{l|}{None}                                                                                                &  &  \\ \cline{1-5}
\multicolumn{1}{|c|}{\multirow{9}{*}{Structure}} & \multicolumn{1}{l|}{None}              & \multicolumn{1}{l|}{\begin{tabular}[c]{@{}l@{}}Bittorrent, Napster,\\ Publius\end{tabular}} & \multicolumn{1}{l|}{\begin{tabular}[c]{@{}l@{}}Kazaa,\\ Morpheus,\\ Gnutella (extension proposals),\\ Edutella\end{tabular}} & \multicolumn{1}{l|}{\begin{tabular}[c]{@{}l@{}}Gnutella,\\ FreeHaven\end{tabular}}                            &  &  \\ \cline{2-5}
\multicolumn{1}{|c|}{}                           & \multicolumn{1}{l|}{In Infrastructure} & \multicolumn{1}{l|}{}                                                           & \multicolumn{1}{l|}{}                                                                                           & \multicolumn{1}{l|}{\begin{tabular}[c]{@{}l@{}}Chord,\\ CAN,\\ Tapestry,\\ Pastry\end{tabular}}                          &  &  \\ \cline{2-5}
\multicolumn{1}{|c|}{}                           & \multicolumn{1}{l|}{In System}         & \multicolumn{1}{l|}{}                                                           & \multicolumn{1}{l|}{}                                                                                           & \multicolumn{1}{l|}{\begin{tabular}[c]{@{}l@{}}Bittorrent (DHT/Trackerless), \\OceanStore,\\ Mnemosyne,\\ Scan, PAST,\\ Kademlia,\\ Tarzan\end{tabular}} &  &  \\ \cline{1-5}
                                                 &                                        &                                                                                 &                                                                                                                 &                                                                                                                          &  &
\end{tabular}
\end{adjustbox}
\end{table}

    Early versions of Gnutella come as a famous example of a decentralized and unstructured architecture, as peers act with equal functions and privileges, and no inherent structure exists on how these peers connect to others, nor does it on storing or retrieving content on the network.
    The bootstrapping method consists on users reading from a set of known Gnutella peers, which is essentially a static list of addresses obtained from a trustworthy source, and attempting to connect to each one of them until a preferred number of known neighbours is reached.
    The unstructured nature of this protocol makes it so there's no systematic way to efficiently retrieve content, as thus peers must flood the network with content queries until either a reply is met or the predefined TTL value is exceeded, as can be seen on Figure \ref{fig:gnutella-flood}.

\todo{reference gnutella-flood.png}

\begin{figure}[!h]
\centering
\includegraphics[scale=0.5]{img/gnutella-flood.png}
\caption{Demonstration of Gnutella's file searching mechanism \cite{p2p-survey-1}}
\label{fig:gnutella-flood}
\end{figure}

    Partially centralized architectures were defined as similar to those which are decentralized, but with the added caveat that some peers are chosen to service a portion of the network.
    This is done to take use from the fact that not all network peers are alike in terms of memory, computational power, or other relevant resources.
    As such, more capable peers are elected as "supernodes" and are delegated with more responsibilities, noting that these self-configure in situations where such supernodes fail or willingly leave the network, and thus there is no single point of failure as there would be on a true centralized architecture.

    A hybrid architecture approach in a P2P network employs some elements from the client-server architecture.
    With Napster as an example, whilst peers still operate as servers or clients, they must contact an intermediary and central server when querying for content, which will in turn redirect them to one or many peers that contain it - a similar concept applies for BitTorrent, where such intermediary servers are called trackers.
    Obviously, the choice to add a centralized aspect to the architecture hinders many of the advantages from a purely unstructured solution - namely its scalability, resilience to failure, and decentralization of control - as a trade-off to facilitate the control and maintenance of the network, as well as the peers' ability to bootstrap to it and locate content.

    A P2P architecture is structured if it employs some non-random and systematic criteria on how the network operates, e.g., how peers organize themselves and where content is stored and how it it retrieved.
    For example, FreeNet uses the content's hash as a key that is used to query for it, and which in turn is used by the peers in each subsequent hop to know where to forward the request, instead of flooding the network in attempts to blindly find it, like Gnutella does.
    Many of the structured P2P architectures rely on distributed hash tables (DHTs), which act as a decentralized map structure that binds a given key to some content in the network, in such a way that the full key-space is partitioned over the peers.
    Two examples of structured P2P architectures that use DHTs can be seen in Figure \ref{fig:dht-usage} - to the left, the Chord algorithm uses a circular DHT where each peer knows the location of some peers that are their predecessors, and some that are their successors.
    When a peer needs to query for some content, it uses its key to firstly search for it locally and, if it doesn't exist, forwards the query to following peers, and the process recursively continues.
    To the right, the content addressable network (CAN) has the key-space mapped to a virtual two dimensional grid, and its area is partitioned to peers considering their geographical location.
    A straight arrow from querying node to providing node represents the routing path that the querying message must travel: A-B-E.

    Employing a systematic way to self-organize and share content is the means to guarantee that a P2P network can be fully decentralized whilst maintaining a desirable level of performance.
    However, the reliance on structure means that it must be maintained, e.g. managing neighbour pointers on Chord or managing area allocations on CAN, and that can be costly or even impossible with high rates of peer churn, i.e., with a sufficiently large rate of peers entering and leaving the network.

\begin{figure}%
\centering
\subfloat[Chord file query mechanism \cite{stoica2003}]{{\includegraphics[width=5cm]{img/chord-lookup.png} }}%
\qquad
\subfloat[Can file query mechanism \cite{p2p-survey-1}]{\includegraphics[width=5cm]{img/can-lookup.png} }%
\caption{Examples of structured P2P query mechanisms that utilize DHTs}
\label{fig:dht-usage}
\end{figure}

\subsection{Effects to the Network Infrastructure}

    Historically, ISPs have deemed P2P traffic as unideal or even undesirable.
    Besides the aforementioned illegality precedent that is tied to P2P applications, the overall properties of P2P networks make them unappealing to support - due to the distributed nature of these types of networks, the overall traffic is less predictable, with the higher upload traffic volume in edge networks requiring infrastructural investments, and the network-agnostic operation mode of P2P applications leads to inefficient and uncooperative network resource usage.

    P2P networks who neither have structure nor a central point of control have to utilize content retrieval methods which are bound to be less efficient than their counterparts.
    However, architectures which fit in these categories mostly do so with a clear purpose - Gnutella's decentralized nature makes it very hard for individual nodes or external entities to regulate what can happen in the network (such as enforce legal actions), and its lack of structure simplifies the architecture and reduces the overall effort to bootstrap to the overlay, making it a good fit for applications with a high peer churn rate.
    Similarly, FreeHaven, an also unstructured and decentralized P2P protocol, has architectural decisions fit a very specific use case, as it "emphasizes distributed, reliable, and anonymous storage over efficient retrieval" \cite{freehaven}.
    The lack of systematic means to efficiently locate content by these P2P architectures means that more ad-hoc methods have to be used, which are less efficient and thus incur in bigger workloads for ISPs - the usage of query flooding by Gnutella and message broadcasting by FreeHaven are examples of this.

    The usage of structure by P2P networks can, as stated before, result in more efficient content and peer location algorithms.
    However, maintaining such structure also requires a chunk of ISP resources, as peers need to periodically update others, as well as react to peers entering and leaving the overlay.
    The usage of key-value mappings with DHTs can also have potential to be ISP unfriendly as the hash function's purpose is to evenly distribute resources over the network - whilst such property is certainly advantageous in certain use cases, doing so removes any  applicational context that exists in the content - for example, grouping resources which belong to the same web page can't be done, as they will be individually hashed and spread out.

    A first point of improvement is optimizations made in the applications themselves to less degrade network resources.
    An example of these can be visualized in Figure \ref{fig:p2p-optimization}.
    To the left, a point of optimization in the Chord system would try to reduce the number of query messages per resource by increasing the number of successors a given node knows.
    That way, the querying node can instead query not for the single successor it knows, but instead by querying for the one who's ID immediately precedes the content's, thus insuring a least number of hops to retrieve the message.
    This thus reduces the total amount of traffic on the network and improves application times.
    To the right, a point of optimization in the Gnutella system could try to tackle the usage of query flooding to locate data, as such flooding grows exponentially and thus intakes a massive toll on network resources.
    A query flooding system would not be as prejudicial if content was equally scattered throughout the overlay and a given content was an amount of hops away which was average to the overlay's diameter.
    However, as concluded by extensive analysis of user traffic on Gnutella, nearly 70\% of users share no files and nearly 50\% of all responses are returned by the top 1\% of sharing hosts \cite{freeriding-gnutella}.
    An attempt to improve this situation was done in \cite{altruistic-gnutella}, which injected a special node that interfaced with the Gnutella protocol and acted as a cache and load balancer.
    Much like the previous example, these optimizations have the benefit of reduced network resource usage and increased application performance.

\todo{Create version of the gnutella-flood picture that shows how flood isn't nearly as bad if caches exist nearby}

\begin{figure}%
\centering
\subfloat[Chord file query mechamism with larger neighbour knowledge\cite{stoica2003}]{{\includegraphics[width=5cm]{img/chord-lookup-efficient.png}}}%
\qquad
\subfloat[Gnutella file query mechanism that utilizes caching]{\includegraphics[width=5cm]{img/gnutella-flood-efficient.png} }%
\caption{Examples of P2P query mechanisms optimized from their counterparts}
\label{fig:p2p-optimization}
\end{figure}

    Regardless of the many ways through which P2P systems can operate, e.g., in regards to structural mechanisms and centralization, and even disregarding potential application optimizations, no classic P2P system operates in full understanding of the underlying network topology, nor with cooperative behavior with ISPs.
    The network-agnostic manner under which they operate results in overlay networks which are layered on top of the underlay where they run, as exemplified in Figure \ref{fig:overlay-underlay} - as P2P applications are network-agnostic, two neighboring peers could exist in completely different contexts on the common network layer - for example, they could either be connected by a single data link or be multiple network provider domains away from each other.

\begin{figure}[!h]
\centering
\includegraphics[scale=3.0]{img/p2p-topology.png}
\caption{Example demonstration of an overlay network and corresponding physical layer.}
\label{fig:overlay-underlay}
\end{figure}

    P2P application's inability to localize traffic is seen as a big problem - as concluded by \cite{liao2014}, ISPs face bottleneck bandwidth pressure in the large scale Internet of the future, in particular from P2P applications, and an increase in P2P users is not necessarily harmful, and perhaps helpful, if traffic locality can be boosted.
    Thus, there is a necessity to ensuring that peers prefer to generate traffic within network locality over traffic which crosses network borders.
    A first step would be increasing the availability of content within network boundaries, through means such as cache injections or peer content serving incentives.
    Given that content resides locally, a second step would be that P2P applications have a network-aware vision of the overlay, which does not happen with classic P2P protocols.
    This issue is particularly damaging in P2P applications whenever neighbours are selected, and whenever a service exists redundantly in the network and a decision must be made in regards to what peer or collection of peers to consume the service from.
    If it indeed is the case that P2P applications are not locality aware, it can easily be seen how this can be an issue: for P2P applications, choosing peers which are not local to the querying peer may result in more time to retrieve the requested contents; for ISPs, bad network resources management can incur in higher operational costs, and may degrade overall network performance in many cases, e.g., by not being able to stop applications from overusing inter-AS links which usually are network bottlenecks (a conventional wisdom demonstrated in \cite{akella}) and which, due to peering agreements with other ISPs, are less desirable to be overused due to peering costs.
    If the P2P application were to attempt to choose the peers that would most effectively serve the querying peer, it could depend on privileged information that the ISP has on the network's properties and current status (such as the inherent network topology, link properties, or scheduled server maintenance).
    Attempting to optimize peer selection without a co-operational channel with ISPs would be sub-optimal as not enough information is known, and could perhaps even be more damaging with the wrong techniques - consider a peer selection algorithm that chooses the peers with lowest RTT of a probing ping message, whilst having no indication on available end-to-end bandwidth.
    Likewise, attributing neighborship via geographical proximity - much like the CAN architecture does - whilst initially seems like a good step in location awareness, may also not be optimal - ISPs may not always prefer geographical proximity in connections, as peers could be very geographically close but residing in different ASs and thus separated by costly links.
    Other peer-selection techniques focus on randomly selecting nodes, which is simple and resilient to peer churn \cite{qin2009}, but as a consequence is sub-optimal on network resource usage as no network consideration exists.
    It is fair to say that no P2P application can act with full ISP consideration without directly cooperating with it, and simple heuristics should be, whenever possible, traded for methods where full cooperation with the underlay is done - that is, if the needs of both layers are being considered.

    Indeed, it is the case that current P2P solutions are ISP-unfriendly.
    More concretely, \cite{isp-p2p-tussle} shares the view that P2P applications and ISPs are in a tussle since P2P applications generate traffic which favours the application's needs whilst ignoring the ISP's, which in turn upsets the ISP's business model.
    To name a few examples, BitTorrent seems to employ peer selection algorithms which do not consider the underlay network, which can result in degraded download performance and increased load on ISPs \cite{qin2009}.
    \cite{karagiannis} found that since this protocol is locality unaware, 70-90\% of existing local content was found to be downloaded from external peers and suggests that locality-aware content distribution algorithms could significantly reduce the total amount of traffic generated.
    Likewise, Gnutella generates traffic which is not ideal, as it may have to cross ISP network boundaries multiple times \cite{estimating-gnutella} due to the same fundamental issue stated before - an application layer that operates in disregard to the network underlay it runs on.

    As \cite{dan-Commag10} describes, the ongoing friction between the overlay and underlay layers has made it to the point where ISPs have chosen to throttle the bandwidth of P2P traffic, or even outright blocking it.
    In return, P2P applications have tried to mask their presence to bypass such restrictions via tunnelling or using non-standard and random port numbers.
    This is an unsustainable system that is bound to hurt both ISP profit and application functionality, and a strategy of cooperation between the overlay and underlay layers is crucial to guarantee that the requirements of both parties are met.


\section{Content Distribution Networks (CDNs)}

\subsection{Concepts and applications}

    A content distribution network (CDN), as the name implies, is a network specifically designed with its main focus on distributing content to a set of end users.
    Its design allows for the alleviation of performance bottlenecks on the Internet generated by client requests, and has been recently been considered a powerful tool as a response to the existing high demand for media content, which has a huge share of the global Internet traffic of today.

    Functionalities of CDNs include \cite{cdn-survey}:

    \begin{itemize}
        \item \textbf{Content outsourcing and distribution:} Replicate content throughout the network into edge servers, which are deployed nearby end users.
            Doing so allows CDN clients to pay for their content to be hosted on these edge servers, and in doing so guaranteeing that their content is quickly accessible by end users.
        \item \textbf{Request redirection:} Direct a content request to the most appropriate edge server at a given time.
            This redirection is done considering relevant network properties, such as geographical locations and current server loads.
        \item \textbf{Content negotiation} Leverage the network's properties to fit the needs of its clients through negotiation.
        \item \textbf{Management:} Manage the distribution network, which includes accounting, monitoring, statistical analysis on content consumption, etc.
            A close management of the distribution network is important for its business model, as, besides being needed for a billing system, allows for a better understanding for the usage patterns of the network, which is helpful information in better engineering the network to most optimally serve content with increased revenue and decreased costs.
    \end{itemize}

    The current focus of CDNs is thus to provide content, e.g., web pages, documents, photos, videos, or media-related streams, with high availability and performance.
    The strategy used by CDNs to guarantee a satisfying quality of experience (QoE) on a global scale is the deployment of content close to the end-user - a CDN contains many nodes which are geographically spread throughout the globe and close to the users they wish to serve, and whenever such users request for content, they are routed to the node which is closest to them (\cite{cookbook}).
    Data replication to servers which are strategically placed closest to end-users, coupled with good means to properly redirect such users to the most attractive edge server, is what allows content to be available more often and more quickly.
    These are undoubtedly attractive features in the world of e-commerce, where user QoE can dictate much of the profit - for example, Akamai, one of the leaders in CDN-related services, ran a research concluding, among other things (\cite{akamai}):

\begin{itemize}
    \item A 100 millisecond slower webpage loading speed can result with a 7\% drop in sales
    \item A 2 seconds slower webpage loading speed can almost double the number of visitors who end up abandoning their carts
    \item 53\% of users who use smartphones to visit web stores won’t make the sale if the webpage takes more than 3 seconds to fully load
    \item 28\% of users won’t return to the same web store if they think it takes too long to load
    \item A 250 millisecond faster loading time proved to keep users from visiting a competitor web store
\end{itemize}{}

    It should then come to no surprise that streaming services such as Netflix and Youtube, who now reach a global scale and whose utility is highly dependant on their high availability and low transmission delay, routinely use CDN solutions.
    More broadly, typical CDN customers include media and Internet advertisement companies, data centers, ISPs, online music retailers, mobile operators, etc \cite{cdn-survey}.
    Indeed, companies that wish to provide a given service in the web and who wish to have global presence routinely partner with companies whose focus is providing content delivery services, with popular examples being Akamai, CloudFlare, or Amazon Cloudfront.
    Coupled with the promise of highly available and quick content retrieval, these companies also couple other attractive services, such as firewalls and DDoS protection.
    The Internet's currently most targeted use for media consumption has made it so CDNs and their providers have an important role in dictating a very considerable percentage of flow of traffic in ISP-owned infrastructures, and as such their study and improvement is quite important, as are the efforts to increase harmonious behaviours between content delivery applications and service providers, with the goal in mind being network resource efficiency to guarantee that ISPs can remain operational and applications can provide a satisfiable user experience.

\subsection{Architecture}

    Figure \ref{fig:cdn-conceptual-architecture} represents a high-level conceptual architecture of a CDN. The true power of a CDN comes from its strategically deployed cluster of replicated server - at a global scale, this implies having those servers geographically dispersed and located in or nearby networks with large content demand.
    The origin server possesses the content that is to be served, and the bootstrapping process has it uploaded into the network, which is afterwards replicated.

\todo{Get higher resolution picture of this}

\begin{figure}[!h]
\centering
\includegraphics[scale=8.0]{img/cdn-architecture.jpg}
\caption{Conceptual architecture of a CDN \cite{cdn-survey}}
\label{fig:cdn-conceptual-architecture}
\end{figure}

    Figure \ref{fig:cdn-request-routing} displays how, conceptually, the request routing functionality of CDNs.
    As can be seen, the request is firstly directed to the origin server, and serves only light, basic content.
    In the situation where heavy static content is requested, the origin server redirects the request to the CDN provider, which utilizes a selection algorithm select the most appropriate edge server to serve the content to the end user.

\begin{figure}[ht]
\centering
\includegraphics[scale=0.5]{img/cdn-request-routing.png}
\caption{Request routing functionality of a CDN}
\label{fig:cdn-request-routing}
\end{figure}

    The Request routing mechanism is the one that informs a given user to a given edge server.
    The prevalent approaches are, according to \cite{wichtlhuber2017}, the following:

\begin{itemize}
    \item \textbf{DNS request routing}: The user must first resolve a domain name to retrieve a content. The CDN's DNS processes such request and, utilizing the user's IP address, historical measurement information and current server loads, responds with the edge server that seems most fitting to provide such content.
    \item \textbf{HTTP request routing}: Content is firstly requested to a nearby proxy server, which in turn answers with an HTTP redirect to be resolved by the client in order to find the content.
        The HTTP requests can occur in subsequent rounds and can also use DNS knowledge when the redirection domain must be resolved.
    \item \textbf{Anycast request routing}: The CDN provider announces an anycast prefix to the network.
        Whenever a router receives multiple announcements to the same prefix incoming from different locations, it chooses one considering some custom criteria, usually being AS hop count and Interior Gateway Protocol (IGP) weight.
        Thus, different routers can have a given anycast address mapped to different hosts, meaning the ones that are most suitable for that particular router.
\end{itemize}{}

    These mechanisms are also discussed in \cite{cdn-survey}, but also add:

\begin{itemize}
    \item \textbf{Global server load balancing (GSLB):} Service nodes, consisting in edge servers and GSLB-enabled web switches are interconnected in a network.
        Individual nodes possess global awareness of the network, meaning the status of each individual server.
        With edge servers having more information on the health and status of all other candidate edge servers, and the web switches acting as authoritative DNS servers, the network can enables the support of a global-wide server load balancing mechanism that intakes dynamic server information.
    \item \textbf{URL rewriting:} The origin server redirects the end user via dynamically altering the host pages' URL links.
    \item \textbf{CDN peering:} An extension of a single CDN network to multiple, connected CDNs which serve content on behalf of others when appropriate.
        This is helpful to extend the domain reachability of a single CDN, increase fault tolerance, and ability achieve better performance with more candidate servers to choose and balance loads from.
\end{itemize}

    It is vital that a CDN possesses a clear view of the network performance inside its domain.
    \cite{cdn-survey} lists important metrics which are used to measure CDN performance:

    \begin{itemize}
        \item Geographical proximity of end users
        \item Path latency
        \item Path packet loss
        \item Path Bandwidth
        \item Path startup time
        \item Path frame rate
        \item Server load
    \end{itemize}

    Means through which these metrics are retrieved by the network include network probing of relevant entities (in particular the end users), traffic monitoring of end user to surrogate server communications, and surrogate server feedback retrieval via application requests or measurement probings \cite{cdn-survey} \cite{akamai-report}.

    Having a clear understanding of CDN performance is important for system management in two fronts - firstly, through performance evaluation, by providing the billing and monitoring modules the verification of how the network is faring at its task of caching and delivering content on behalf of clients; secondly, through performance optimization, by providing the logical layer an updated view on network status, serving as contextual input that the system uses to better reason on how to act - for example, in regards to the caching and request routing mechanisms.

\subsection{Effects to the Network Infrastructure}

    As previously discussed, CDNs came as a tool to strategically position content on the network in such a way that it can more quickly and more reliably be retrieved by an end-user.
    These CDN systems are then inherently a means of optimizing network resource utilization in the application layer, and thus are of great interest for ISPs as their mode of operation, if done properly, can be very appealing not only to them but also to end-users that use applications leveraging these networks.

    The usage of a single content-providing server (or a limited set of them) which is far away from the content supply that in turn can have a large geographical distribution is prone to overloading such server and path congestion if a big enough scale is achieved.
    The usage of data caches is a classic solution for network inefficiency problems, which is used by CDNs as a means to replicate content to strategic locations to better serve users, with the added benefit for ISPs that their network resources are efficiently used, with the ability of reducing the total amount of used bandwidth needed for a service to operate - as data travels a shortened total amount of network hops from data source to points of data demand - and reducing congestion of inter-domain links - as data caches will reside locally and redistribute traffic away from highly shared network links.
    It can thus be stated that the relationship between CDNs and ISPs is a win-win situation because efficient network usage has consequentially better service quality.

    However, there does not seem to exist proper cohesion with the underlying network infrastructure by CDNs during application operations.
    As stated by Akamai, a leader in Content Delivery Network services, in their report \cite{akamai-report}, the large scale and complexity of the Internet, where it takes well over 650 networks to reach 90\% of all access traffic, adds to it many challenges to the CDN's role of content delivery - in particular, whilst good investment seems to exist at the first and last mile of internet access (website hosting and end users, respectively), there seems to be little economic incentive to invest in the middle mile, composed of peering points shared among networks, which in turn become network bottlenecks and become susceptible to cause increased traffic packet loss and latency, making inter-network communications unreliable - and loose coordination between autonomous networks with internal biases is pointed as a main cause.
    Due to this, even a well provisioned data center will be at the mercy of the various inter-network bottlenecks that may arrive, and performance takes a hit.
    In fact, the paper suggests a clean-slate redesign of the Internet as a potential solution to its many problems - besides the peering point congestion mentioned above, inefficient routing protocols, unreliable networks, inefficient communications protocols, and application limitations also add to the problematic - but such a redesign to a massive and highly invested global infrastructure doesn't seem plausible.

    In alternative to proper network infrastructural insight, CDNs have to rely on network probing, traffic monitoring, and server feedback, as seen above.
    Even assuming that these are sufficient, the usage of probing techniques will incur in overhead traffic on the network, and even worse if other overlay networks do so as well, in a redundant fashion.
    Similarly, traffic monitoring to extract end-to-end path metrics to end users requires resources and takes time, and may also incur in redundant operations if other overlay networks are utilizing the same strategy.
    Regarding this topic, projects such as IDMaps \cite{IDMaps} and GNP \cite{GNP} described architectures for a global distance estimation service, leveraging measurements made by specialized nodes that retrieve raw network data, and heuristics provide scalable and functionally reliable path costs in metrics such as bandwidth and latency.
    These then consist on systems that centralize and share network probing results into querying entities, thus minimizing overhead traffic on the network, especially between popular Internet points.
    Such advantage goes outside of the CDN realm, being thus useful for any overlay-residing application that wants to utilize network probing to be more underway-aware.

    Advantageous as network probing and traffic monitoring mechanisms can be for CDNs to proper conduct request routing and caching decisions, a case must me made against proper application-infrastructure synergy during decision making in the overlay.
    Much like P2P systems, to infer on network status by measuring it - versus receiving input by trustworthy, authoritative, entities that possess privileged network information, such as ISP administrators - is insufficient.
    Attributing edge servers to end-users entirely on geographical data was previously discussed as being a non-optimal way of assessing node selection at the application layer.
    Whilst it may be intuitive that the best edge server to serve an end-user would be the one most geographically close to it, that is not always necessarily the case, much like was discussed in similar strategies used in P2P systems.
    Again, much like in the scenario of peer selection in P2P systems, the usage of network measurements made by the CDN itself to better pick the appropriate end-server, while it could potentially be beneficial, it could certainly be improved if it used additional, hard to retrieve data that only ISPs or other privileged entities could possess, and which are at a position to guide applications in the infrastructure with whom they have detailed knowledge.
    There indeed seems to be a consequential coupling between overlay decisions in the CDN systems and the underlying infrastructure - do the CDNs not take ISP input when redirecting clients, suboptimal choices can be made that would be prone for bottleneck congestion, and would ISPs take all the responsibility in redirection, user-level application QoS agreements might not be met.
        \cite{pushing-cdn-isp-collaboration} states that this lack of awareness to network status is indeed problematic for CDN systems, listing end-user mismatching to edge servers based on dubious DNS-based location binning and resource consuming, non exact methods to detect bottlenecks, as well as lack of agility in server deployment in ideal locations.
        This is a view shared by \cite{cdn-isp-cooperations}, which adds that these problems reside in a share medium that raises the opportunity for cooperative behavior that would enable better application performance and optimized ISP resource utilization.
        In fact, Akamai themselves have formed content delivery strategic alliances with major ISPs, such as AT\&T, Orange, Swisscom and KT \cite{pushing-cdn-isp-collaboration}, which hints at this type of partnership being the norm for content distribution technologies.

        ISP input permits applications to act in a more network-aware fashion than without it - whereas pairing overlay nodes or deploying edge servers in terms of geographic distance, RTT distance, or any other metric, may give further decision power than a purely network-agnostic overlay system, proper guidance by ISPs could help pairing based on more complex metrics that, besides the aforementioned ones, also consider ISP objectives - for example, to minimize network distance, avoid bottlenecks, locate content caches, etc.
        This is further heightened if many other overlays coordinate their efforts with the ISP, which can now in turn orchestrate its traffic, which would previously be generated with no input from it, in such a way that maximizes network resource utilization and, consequently, application performance - this is an important point to consider because one-sided application decisions cannot fully grasp network status nor can they coordinate efforts with other applications for a more harmonious, and thus efficient, Internet.


    \section{Server mirroring}

    \textbf{[Change the section 'server mirroring' and talk about the classic server-client architecture instead, and in it bring up the server mirroring technique.]}
    \todo{finish this}

    \subsection{Concepts and applications}

        Server mirroring is the act of continuously replicating a server into another, essentially creating an exact copy of it that is now accessible as if it were the original.
        Whilst CDNs aim at replicating chunks of contents wherever it may be necessary, the act of server mirroring performs an integral copy of a server which is self sufficient at serving a given client, as long as it periodically checks up with the primary server for synchronization \todo{cite}.
        It is a standard business strategy that uses redundancy as a means to increase reliability, availability and performance.
        The existence of many servers that perform the same task means that these can be strategically chosen to serve a client in a given situation, e.g., by selecting the one that has small end-to-end message delay and little current server load.
        Figure \ref{fig:mint-mirrors} shows an application of this, where multiple server mirrors exist to deliver software packages to the Linux Mint distribution. The user has the choice to manually select one of these mirrors, and ideally chooses the one that is most fitting to them.

    \begin{figure}[!h]
    \centering
    \includegraphics[scale=1.0]{mint-mirror.png}
    \caption{Linux Mint prompt to select a software repository mirror}
    \label{fig:mint-mirrors}
    \end{figure}

    \subsection{Effects to network infrastructure}

        Much like the content replication utilized in CDNs, an integral replication of a main server proves itself as an advantageous tool capable of delivering services more closely to users, and as such allows the reduction of total amount of bandwidth used to serve all clients.
        Much like all previous use cases discussed before, optimizing application traffic is crucial to guarantee good network resource usage, and in case of server mirroring it comes down to good strategical deployment and dynamic, intelligent algorithms to properly attribute mirrors to requesting end-users.
        Giving end-users the choice to manually select the serving mirror seems problematic, as application-generated traffic is not optimized.
        In fact, considering the Linux Mint software package distribution discussed above, despite currently existing seventy mirrors deployed throughout the globe to fit this role, a large number of these remains mostly unused whilst the main and default server is constantly prone to overworking \cite{mint-article}.
        It can be stated that end-users both don't care enough to optimize traffic nor do they have enough information to properly do so even if they did.
        Deployment of server mirrors is a great tool that brings with it the issue of optimizing server selection, and much like all examples given so far, traffic generated by applications can be firstly optimized by the applications themselves if they consider static and dynamic information of the network they operate on.


    \section{Edge Computing}
    \todo{finish this}

    \textbf{[Like the previous sections, give an introductory overview of what edge computing is, how it has gained popularity in the present because of the shift to the cloud, and the challenged they face in proper selection of where to place edge services that better fit the needs of the client.]}

    \section{Traffic optimization by applications and layer-cooperative approaches}

        This section serves to display the proposed solutions and existing implementations that have been made in the attempt to optimize application traffic utilizing network information.
        Given the increasing scale of the Internet as a near ubiquitous system, and increasing tension between service providers and applications, it comes as no surprise that the area of layer cooperation has been through exhaustive work.
        Many solutions have been devised for specific use cases, with varying degrees of power to each one of the layers.

        \subsection{Peer-to-peer applications}

        Many different mechanisms have been developed with the goal of decreasing tensions between ISPs and P2P applications, which is a subset of the general layer cooperation problem.
        Figure \ref{fig:p2p-isp-interactions} represents a grouping proposed by \cite{dan-Commag10} where such mechanisms are ordered in agreement with how much involvement the P2P systems and ISPs have. These classes are as follows:

    \begin{itemize}
        \item \textbf{Class 1}:
            There is not much interference in the overlay by ISPs nor are P2P systems cooperative.
            Instead, ISPs apply traffic engineering methods to selectively favour types of traffic.
            This is usually done to guarantee certain QoS levels to some classes of traffic, which are then to be treated favourably at the forwarding and routing levels.
            Examples of such techniques are DiffServ, Multi-Topology Routing (MTR) and MultiProtocol Label Switching (MPLS).
            These classes of methods do not fix the underlying application behaviour, but are instead used to control preexisting traffic.
            As such, the peers' routing decisions are not affected and P2P traffic still remains non localized.
        \item \textbf{Class 2}:
            There is ISP intervention in the overlay in such a way that peers continue normal operation without realizing that such interventions occur.
            This can be reached via the use of proxies that can affect the control plane with the redirection of content requests to local peers, or at the data plane with content caches which act as normal peers and are strategically placed in the network.
            These methods are advantageous because they do not require any changes to P2P protocols, because the ISP has an active role in molding to the overlay, intercept traffic, and either help or guide it in a way that favours them.
            Indeed, these techniques can be proven to work , as concluded in \cite{dan-Commag10}, and put into practice, for example, in \cite{programmable-trackers} and \cite{configurable-trackers} via the specification of a BitTorrent tracker that is programmable to allow for P2P qualitative differentiation and ISP-cooperative traffic engineering that could help reduce inter-domain traffic significantly.
            Additionally, in \cite{freeriding-gnutella} with the injection of special nodes on the Gnutella overlay which interface with the base protocol nodes but with the added caching and load-balancing mechanisms.
            This seemed necessary after concluding with extensive analysis of user traffic that nearly 70\% of users share no files and nearly 50\% of all responses are returned by the top 1\% of sharing hosts \cite{freeriding-gnutella}, and such a solution  helps minimize the total size of query floods and more evenly distribute content on the network for decreased network resources usage.
            However, this class of mechanisms are not without their challenges - firstly, it involves much effort by ISPs, as it requires structural upgrades and constant adaptiveness to new and changing P2P protocols.
            Perhaps worse, even considering proper budget and maintenance, such methods can prove themselves to be not possible at all - for legal reasons, as data caches could possibly contain illegal content; and for technical reasons, since the packet inspection required by ISPs to detect and steer P2P traffic may be blocked due to the peer's attempts to mask its traffic.
        \item \textbf{Class 3}:
            Relative to previous classes, the active role is switched and it is the P2P system itself that acts in regards to the underlay it operates on, but without ISP involvement.
            Peers probe the neighboring network elements as a way to get more familiar to connection properties, and act on these probings during operation, e.g., when choosing neighbours to construct the overlay network with, or when choosing to whom request a given resource.
            Whilst these methods can be advantageous for both applications and ISPs, it can't be assumed that to always be the case - as peers have no ISP input, they cannot have a full scope on the network and ISP needs, and as such these application optimizations can end up being more hurtful than helpful.
            For example, consider a scenario where a peer uses RTT measurements to choose between two candidate peers, but the one that is geographically closest to it belongs to another AS, and his preferring it for content supply would incur in more costs.
            The paper describes this class as a "win-not-lose" situation, meaning that while the P2P system can, in the right circumstances, improve their performance via measurement-oriented strategies, the ability to act beneficially to the underlay without any feedback from ISPs cannot be guaranteed.
            Such an example of class 3 mechanisms could be seen in \cite{qin2009}, which improved BitTorrent's download performance and even managed to reduce ISPs' backbone and cross-ISP traffic.
            The technique consisted in having peers send traceroute measurements to the tracker, which in turn grouped them into local, intra-ISP and inter-ISP groups, with the assumption that inter-ISP links generally have much more latency than the rest.
            As peers would later query the tracker for content, the returned peer list would be biased in such a way that promotes traffic locality.
            Another example of this is \cite{kim2011}, which devised a CDN-P2P hybrid where peers utilize RTT measurements to group themselves by separate orders of geographical proximity with the same intent of the previous example, which is to localize traffic whenever possible.
            This technique also proved itself to be advantageous, as the solution was more efficient in terms of total service disruption time when compared to a previous iteration of the hybrid architecture which used random peer selection to look up available target peers.
            As a final example, \cite{topology-aware-p2p-server-selection} proposed a node binning scheme that groups nodes of similar orders of magnitude of RTT values to pre-defined landmarks, and utilized such scheme for topology-aware overlay construction mechanisms in some unstructured and structured P2P overlays.
            Results allowed to conclude that even surface levels of relative topological distance were advantageous and can significantly improve application performance.

        \item \textbf{Class 4}:
            Full and active cooperation exists between the ISPs and P2P systems.
            The role of the ISPs is to provide information and guidance, and P2P systems let themselves be influenced during operation.
            It is the methodology that most comes close to a mutually advantageous scenario for both parties, given that they both keep the entire group's needs in mind.
            For example, \cite{locality-aware-p2p} proposes an oracle that receives as input from a querying peer a list of candidate peers, and ranks them in order of proximity to that peer; such method was tested in simulation and proven to decrease negotiation traffic and improve scalability of P2P networks.
            Similarly, \cite{configurable-trackers} proposes a framework for programmable trackers in a BitTorrent trackers scenario, providing an interface to directly define tracker behavior which, given ISP input, can provide a collaborative scenario between ISPs and the BitTorrent users in an attempt to, among other things, localize traffic.
            The functional intent is that the oracle possesses privileged network information and acts on it to provide guidance to querying applications, and thus has the liberty to impose policies and optimizations, e.g., pair peers which are the least number of network hops apart via a Dijkstra algorithm.
            Another more complex approach that could be used by the oracle proposed by \cite{han2009}, which contains algorithms to dictate peer selection, task assignment and rate allocation.
            The method requires the full network topology as input - including link capacities and peer service costs - to minimize file downloading time and cost.
            The oracle would also be free to enforce ISP biases as its preferences by modifying such algorithms as to, for example, minimize usage of costly links (such as inter-AS ones).
            The ALTO working group - whose work this thesis attempts to materialize into a working system and further extend its features - was formed to standardize the oracle-user scenario so it could be properly used in many situations at the scale of the Internet.

    \end{itemize}

    \begin{figure}[!h]
    \centering
    \includegraphics[scale=0.65]{img/approaches-isp-p2p.png}
    \caption{Existing approaches to decrease tension between P2P applications and ISPs (\cite{dan-Commag10})}
    \label{fig:p2p-isp-interactions}
    \end{figure}

    \subsection{Content Distribution Networks}

        Given the current share that CDNs have on global Internet traffic of today, coupled with the demand for a good QoE by end-users, it should come to no surprise that this application domain has also been through efforts to optimize its traffic.
        One such way to do so is to optimize client query redirection, i.e., better choose which edge server should be attributed to an end-user when a name resolution is requested for some content.

        \cite{gromov2014} considers a CDN built to deliver video data where some given set of content exists redundantly in many edge-servers, and argues for an algorithm where the choice is made to optimize client download time, which in turn has to consider the network parameters at time of request, as well as current server load.

        Some simple, flexible and scalable techniques exist that utilize no ISP input.
        For example, \cite{topology-aware-p2p-server-selection}, mentioned previously for its P2P overlay construction with a binning technique based on RTT to landmarks, also utilized such binning technique for improved server selection.
        Similarly, IETF tackled application traffic optimization via multi-CDN cooperation, and devised a problem statement in regards to Content Distribution Network Interconnection (CDNI) \cite{cdni-problem-statement}, which outlines the efforts needed to specify a set of interfaces that allow for the interconnections of many CDNs, with the added benefits that a multi-CDN system, over an individualistic one, will have better properties, i.e., availability, coverage, and capabilities, as well as better QoE for the end user, and reduced delivery costs.
        The four devised interfaces (CDNI Control interface, CDNI Request Routing interface, CDNI Metadata interface, and CDNI Logging interface) are all control plane interfaces to be operated at the application layer, and the group states that no new application protocol needs to be devised, and instead existing ones could be leveraged, e.g., HTTP, Atom publishing protocol, XMPP, and in particular to the CDNI Request Routing interface, the ALTO protocol who's the focus of this work.

        \cite{cdn-isp-cooperations} argues for the advantage of cdn-isp cooperative interactions and overviews three relevant strategies that will be now discussed briefly: Provider-aided Distance Information System (PaDIS) \cite{pfa-10} is a system deployed and controlled by ISPs that monitors the network by listening to EGP and IGP messages and contains a privileged view of the topology and its status, and provides a service that ranks host-client pairs in regards to, for example, delay, bandwidth, or hop count, and experimental testings concluded that download times of content provided by CDNs that utilize PaDIS can be improved to a factor of four and gives much flexibility for ISPs for traffic engineering; Content-aware Traffic Engineering (CaTE) \cite{fps-12} is designed in a similar manner to PaDIS but requires no client-side configuration, and experimental results concluded in network wide traffic reduction by 15\%, decrease in ISP link utilization by 40\% and increased user-server performance; Network Platform as a Service (NetPaaS) \cite{fpl-13} was devised to fulfill two key enablers in a fruitful CDN-ISP collaboration - user-server assignment, as it was tackled in the previous two examples, and server allocation, i.e. where should a CDN deploy its servers and its contents, thus having increased functionality over the previous two examples, and concluding that, beyond the common advantages of increased application performance and better ISP traffic control over network resource utilizations, this system facilitates the task of server allocation for CDNs, reinforcing the discussed advantages and simplifying the task for CDNs.
        Still in the topic of CDN's edge server selection, \cite{wichtlhuber2017} suggests a way of optimizing anycast request routing, which differs from the DNS-oriented request routing techniques, which, while very light in terms of network engineering and infrastructural overhead when compared to existing alternatives whilst maintaining a close to optimum network path, it sacrifices flexibility to do so, as it is agnostic to the network's status and not much network engineering can thus be done.
        As such, the work proposes anycast request routing utilizing software defined networking (SDN), where load balancing is made at the ISP network with help of CDN-provided additional information.
        This example of layer cooperation can allow for many optimization opportunities that leverage an existing and low-maintenance mean of request routing with the flexibility achieved with SDN solutions.

        \subsection{Server-client applications}

        Attempting to optimize web server selection, \cite{kenichi} argues that DNS-oriented solutions, which select the nearest server but also employing load balancing, may not be the best at optimizing server-client QoS levels.
        Instead, it is proposed that selection is based on QoS measurements, from which three types are distinguished: a static method, such as choice based on least number of hops to server (which is unlikely to change); a dynamic method, consisting on dynamic instantaneous probing of the network to monitor, for example, round-trip time (RTT) delay to the servers; and statistical methods, which decide based on a larger set of measurements made in various points in time.
        Utilizing the latter method, RTT measurements and web-related request benchmarking is made, such as time to establish TCP connection, elapsed time from GET HTTP method to first packet received, time to retrieve data fully, etc, every five minutes and spanning several weeks.
        The work concluded that statistical methods used to select between multiple equal web servers had high correlation with download time from the selected server, but optimizations should be evaluated in regards to computer workload and the amount of probing traffic.
        Tackling a similar challenge, \cite{swain} proposes a method of server mirror selection which is better optimized than the more popular approach of giving the user the selecting choice.
        The proposed solution's architecture consists of two types of agents: a client agent, which monitors the mirror server it was deployed in and stores static information, e.g., geographical location of server and maximum capacity, and dynamic information, e.g., current load and bandwidth.
        This information is then sent to the other role of the architecture, the server agent, which compiles it and acts as an oracle that is queried by users whenever mirror selection is needed, replying with a ranking of candidate servers based on bayesian networks.

        Congruent to the task of optimizing network traffic with layer cooperation, \cite{adaptable-overlay} proposes a reconfigurable and adaptable overlay multicast system, further optimizing the multicast strategy - used for group communication as a means to reduce redundant traffic - and leveraging collaborative efforts between it and the ISPs to construct multicast distribution trees whilst integrating traffic engineering mechanisms for the task of network usage optimization.

        \subsection{Edge Computing}

        \textbf{[ALTO has a solution for server footprinting and connectivity that gives users more information about the cloud, and thus lets them better decide how to better rent servers for edge computing]}

        \todo{Lookup non ALTO solutions that aim to bridge communications between ISPs and users in the task for edge computing}
        \todo{Finish this}

        \subsection{Summary}

        Concluding, application traffic optimization does indeed to be a common concern for P2P, CDN, and Server-Client systems, as it improves application performance.
        Indeed, potential to improve exists if attempts are made to better comprehend current network status to aid application decisions, and so is made by realizing more about current network status, whether by probing it, or retrieving that information from - or delegating decisions to - authoritative sources.
        On the other hand, it does seem to be the case that similar improvements occur with higher cooperation with ISPs, without the need to redundantly probe the network for information that will be vastly more limited that the overall insight that the ISP itself can provide, which could do so in the exchange of increased traffic engineering potential, resulting in a win-win scenario that is sustainable for both sides.

    \section{Application-Layer Traffic Optimization (ALTO) working group}

    \subsection{Context and Motivation}

        Acting on research indications that improved peer selection algorithms based on ISP-provided information could help reduce ISP costs and increase P2P application performance, the Internet Engineering Task Force (IETF) devised working groups to explore IETF standardization in the area of layer-cooperation \cite{seedorf2009}.
        Among them is the Application-Layer Traffic Optimization (ALTO) working group, whose domain is traffic localization.


        This ALTO working group designed an HTTP-based protocol whose function is to allow hosts to query privileged servers on network information.
        The IETF-devised working group's project has gathered much academic interest \cite{seedorf2009} \cite{provider-aided-cdn} \cite{sampaio2018} \cite{liao2014} \cite{dan-Commag10} and has been suggested as an appropriate framework for various problems \cite{fps-12} \cite{pfa-10} \cite{wichtlhuber2017} \cite{cdni-problem-statement}, to name a few, and this in of itself is also a small subset of a larger current preoccupation in the underlay-overlay tussle and the attempt to find means of collaborative layer effort.
        The envisioned scenario of the service provided by the ALTO architecture, as can be seen on Figure \ref{fig:alto-design}, considers both the physical and application domains - the underlay and overlay, respectively.
        The ALTO service is provided by some oracle, which needs to be himself informed on network information that can take many forms - topological structure, routing costs, static policies, etc - and, most importantly, such data is to be fed by an ISP or such other authoritative entity that contains truthful and relevant network information that the oracle could deem useful in aiding its clients.
        Using Figure \ref{fig:alto-design} as an example, consider that "Peer 2" wishes to retrieve a given resource from the overlay, and after querying for its whereabouts - via querying a tracker, deploying query floods on the overlay, or some of the means utilized by structured P2P networks discussed above - the peer is aware that the resource resides both on "Peer 1" and "Peer 3".
        Aware of the fact that choosing whom to consume a service from has impacts on both application performance and network resource utilization, "Peer 2" is to use the ALTO service, querying the oracle on information pertaining to the candidate peers, and in regards to metrics that better fit the needs of the application (because different applications could have different QoS metric priorities in mind, such as a media stream with low delay needs or a file sharing application with focus on bandwidth).
        The ISP is then be in full control of engineering how the traffic from this resource transfer will flow, and can steer "Peer 2" in favoring "Peer 3" - since they reside in the same network this would improve network resourcefulness and there would be no need to make use of peering link to an external network.
        As could be deduced from this and similar scenarios, an architecture containing one or more servers that are knowledgeable on the network they reside on could be an important tool to make P2P applications locality-aware, a common goal for the underlay and overlay parties since it is a win-win scenario.

    \begin{figure}[!h]
    \centering
    \includegraphics[scale=0.75]{alto-design.png}
    \caption{ALTO scenario \cite{seedorf2009}}
    \label{fig:alto-design}
    \end{figure}

        Despite its origins lying in the efforts to localize P2P applications, the ALTO protocol and encompassing system is now being considered in other fields, to be now further discussed.

        A first area of interest is CDNs, most specifically the current works in specifying the CDNI Request Routing Footprint \& Capabilities Advertisement interface \cite{alto-cdni(draft)}, which is a subset of the CDNI standard \cite{cdni-problem-statement} that aims to allow upstream CDNs to query known downstream CDNs if they are able and willing to accept the content request.
        In particular, one of the main functionalities of the CDNI request routing interface is the ability for upstream CDNs to retrieve static or dynamic information on download CDNs (resources, footprint, load), which they provide themselves, and that allow upstream CDNs to better choose the appropriate edge server that could serve a given end-user.
        ALTO serves as a good protocol to implement such functionality because it fits its use-case: some node (in the upstream CDN, where the content query originates) wishes to improve its routing (in regards to resolving content requests) by using information that is hard to deduce by itself to properly choose the most efficient node (the downstream CDNs where the content resides).
        At a more abstract level, this is similar to the use case fulfilled to P2P applications to help them better select peer connections.

        Edge computing, similarly to CDNs, use a paradigm of flexible service distribution that enables deployment closer to the end user for better performance, and thus is inherently effected by network status.
        Current work is being made on the benefits of using ALTO for the proliferation of network information to aid the task of deployment of functions or applications in the network edge \cite{alto-determining-service-edge(draft)}.
        Much like the previous example, ALTO is being used to guide an application in a decision with application and network resourcefulness impacts - by querying the ALTO server, the client can query for information that regards to points of presence (PoP) where functions/applications can be deployed, such as cloud computing provider's available resources, e.g., CPU, RAM, or storage, but also network information that pertains to the outside of the PoP, mainly network connectivity metrics, e.g., end-to-end bandwidth and delay, and routing costs.
        The utilization of the ALTO protocol in this context would allow edge service clients and providers, as well as ISPs, to combine efforts as a means to optimize edge computing deployment that considers the current network status, and doing so would thus result in benefits for both end-users and infrastructure maintainers.

        More broadly, current works are also being done in specifying abstract network entities path arrays between points \cite{alto-path-vector(draft)} and time-sensitive cost values \cite{alto-calendar-cost-map(draft)}, both of which share higher insight of the network, at the discretion of the ISP, as a means to provide even more context to applications about the infrastructure, such as identifying potential path bottlenecks and high traffic time peaks, respectively, and thus improve the ability of applications to properly generate application traffic on the network.

        A mode of operation where applications no longer operate in disregard to the network infrastructure they run on, but instead in deep consideration of it, could help significantly alleviate the issues emerging from the tension between the underlay and overlay, and is of mutual interest - improving application performance and reducing infrastructural costs.
        Enabling a communication channel can thus allow for many different co-operational use cases besides the aforementioned ones - for example, redirecting users to nearby data caches or warning them of server maintenance ahead of time.
        The existence of an all-encompassing oracle could also prove beneficial for applications which utilize periodic network probing to guide their choices, as such information could be measured by a select few nodes in the network and applicable to all nodes which are close-by to such node in ways that the ISP deems advantageous (such as belonging to the same AS or geographically near), thus minimizing the amount of probing used.
        ISP-guided counseling, however, does more than centralize measurements that could be done by the application itself - besides also containing measurements that could only be retrieved by the ISP due to its privileged access to the network (such as IGP packet inspections or secure SNMP queries), by handing over the decision-making process to it, the ISP is in position to better steer traffic in a way that favors internal policies, such as peering agreements, current traffic flow of other applications, known bottlenecks, etc, that could not be deduced by the applications alone.
        Thus, in the decision of how to generate application traffic, the responsibility should reside in both application and infrastructure as a way to benefit all relevant parties - the end users, the application stakeholders, and the service providers - and the ALTO protocol is an enabler of a mutually cooperative layer interaction mechanism that, by becoming the standard, would aid towards a sustainable life-cycle of the Internet.

        Finally, standardizing an architecture and related protocols for a clear problem domain could help a large subset of similar issues - a well defined and tested specification would exist, thus allowing many applications to leverage the ALTO protocol's functionalities to their needs, not requiring further cycles of development for a specification when one already exists.
        Also, the attempt to standardize the oracle pattern is helpful as it joins forces from many different domains which share common problems (many exemplified previously) into a single specification that interfaces with ISP knowledge, and would target issues such as security and scalability, creating a single point of convergence that is mature enough to be adopted by ISPs and applications, accelerating the transformation of the Internet as individual players would not need to develop their own specification.

    \subsection{Architecture}

        The high-level conceptual ALTO architecture can be seen on Figure \ref{fig:alto-architecture}.
        Central to the operation is the ALTO server, which stores network information and provides it to querying clients.
        Such network information is provided by trustworthy and relevant entities, such as information derived by routing protocols, ISP-specific policies, network historic measurements, and network feedback provided by third parties regarding application performance on the network.
        Two protocols can be seen as part of the general architecture: the provisioning protocol - not currently contemplated by the ALTO working group - should specify how information is provided to the ALTO Server; the ALTO protocol - main focus of the working group with the same name - specifies server-client interactions as a request-response interface for retrieval of network attributes.
        The ALTO client is the main consumer of the ALTO service as a whole, and it queries the ALTO server on network information whenever it deems such data as necessary to what it's doing at a given moment, with some potential use cases discussed previously.
        An ALTO client could be seen as any entity which is able to interface with the ALTO protocol with the role of a client, and as such is not tied to a specific implementation - in the example of P2P file sharing, a peer can act as an ALTO client (like the example scenario in Figure \ref{fig:alto-design}), but instead a tracker could enhance its role in assisting peer communication by having an embedded ALTO client which would then act on behalf of querying peers as to provide them with an optimal response, minimizing needed protocol modifications and thus facilitating integration with currently existing tracker-oriented P2P solutions.

    \begin{figure}[!t]
    \centering
    \includegraphics[scale=0.60]{alto-architecture.jpeg}
    \caption{ALTO architecture (adapted from \cite{alto-protocol}) }
    \label{fig:alto-architecture}
    \end{figure}

    \newpage

        The ALTO services contemplated by the corresponding working group can by visualized in Figure \ref{fig:alto-services}.

    \begin{figure}[!h]
    \centering
    \includegraphics[scale=0.55]{alto-services.jpeg}
    \caption{ALTO services (adapted from \cite{alto-protocol}) }
    \label{fig:alto-services}
    \end{figure}

    The ALTO server stores and provides special mappings in the form of network and cost maps.

    A network map provides network location grouping identifiers and the corresponding aggregated endpoints.
    It utilizes Provider-Defined Identifiers (PIDs) as a key, and the mapping itself is left to the responsibility of the providers - it is thus a way of indicating that many endpoints should be handled similarly.
    A provider can then aggregate endpoints by geographical proximity, one or many subnets, one or many Autonomous Systems (ASs), etc., and attribute properties to the aggregate, instead of the endpoint.
    This is advantageous not only for scalability reasons - since it can compress information - but also because it allows ISPs to abstract network endpoints into groups, thus ensuring privacy of network topology details whilst maintaining useful network guidance, as the ISP has full control of how endpoints are aggregated, and consequentially how traffic is engineered since this changes how clients interpret resources.

    A second resource type provided is a cost map, which can be defined as a matrix M, where $M_{ij}$ - with i and j being the source and destination PIDs, respectively - is the associated path cost between the two indexes.
    The cost has two components: its metric and mode.
    The ALTO base protocol only defines a single, generic, cost metric called "routingcost".
    However, \cite{alto-metrics} is currently specifying more concrete metrics, with many associated with Quality of Service (QoS) evaluation, e.g., one way and round trip delay, packet loss and throughput.
    The other cost property, cost mode, can either specify that the metric is to be interpreted as a numerical value or as an ordinal ranking among all other costs in that cost map - this is useful in cases where too much network information is not deemed reasonable to share, and a simple order of preference that doesn't expose too much infrastructural detail can suffice.
    The decision to separate network and cost map information into two types of resources comes from the reasoning that network mappings are unlikely to change, whereas cost mappings could be periodically updated.
    As such, it alleviates client applications from the need to retrieve redundant information, and the ability to only retrieve a subset of it - this ability is further expanded in the map filtering service, which allows an ALTO client to further specify which regions of the requesting maps it wishes to retrieve (much like a "SELECT" statement from the Structured Query Language (SQL)), and only these are transmitted to it.

    Finally, the last two services focus on mappings that regard to specific endpoints, instead of abstract mappings that utilize PIDs.
    An endpoint is currently identified by one of the following: IP address, MAC address, or generic overlay ID.
    The endpoint property service maps to an endpoint a set of properties, e.g., geographical location or connectivity type, and the endpoint cost map has the same meaning of a cost map, but mapping to particular endpoints and not abstract collections.
    The ISP has thus the ability to work with abstract aggregates or specific endpoints, showing as little or as much network information as it deems fit.

    As could be seen, the ALTO project specifies an architecture for trading of network-related information, with well defined roles and a request-response protocol to fulfill interactions between them.
    It also attempts to standardize such interactions in the form of data structures with well defined attributes which are then to be manipulated for each use case.
    This could then serve as a useful service for any application that wishes to retrieve network information that could improve its decision making at the application level.
    It is important to note that there are restrictions to what kinds of information are contemplated by the ALTO protocol - for example, transport-level congestion is beyond its scope, and thus should not replace conventional mechanisms.
    The type of data which is valid to consider, according to \cite{alto-problem-statement}, should not be easily obtainable by the clients themselves (such as end-to-end delay), and should be variable on a longer timescale than the instantaneous kinds that are seen on, for example, congestion control mechanisms, as the frequently resulting querying traffic would be counterproductive to the task of traffic optimization.
    Potentially valuable information that is in the ALTO scope would then have to be harder to obtain without aid of this service, and not highly mutable through time - for example, routing costs, geographical locations, network proximity, operator's policies, scheduled down-times, etc.

    This project is, at time of writing, still on-working, with many drafts being created and updated, as the ALTO project matures and increased its domain applicability.
    These are, however, relating to service extension and deployment, since the main architecture, protocol design, implementation guidelines and security analysis are fully published into their respective RFC documents, serving as pillars for this work, and the ongoing efforts will serve as inspirations for potential extensibility.

\subsection{Viability}

\subsubsection{Security}
\label{sssec:alto-security}

    Given the nature of this system - the trading of sensitive network structure information that can alter application behavior - it is quite apparent that its design and implementation are not without challenges from a security perspective.
    \cite{alto-problem-statement} \cite{alto-protocol} \cite{alto-deployment-considerations} include discussions by the working group regarding security preoccupations at the development and deployment stages of the system.

    Utilizing the "STRIDE" threat model \todo{cite}, the main threats to the ALTO architecture can be summarized as follows:

\begin{itemize}
    \item Spoofing of a legitimate ALTO server that would mislead clients with wrong information - this could give the malicious party the ability to change traffic to its will.
        Spoofing of the clients themselves can also occur, and could allow a malicious party to retrieve sensitive network data outside their permission.
        Finally, spoofing of a provider of network status that could then feed information into the server to be spread into applications, possibly misleading them in the same way an ALTO server spoofing could, by proxy.
    \item Tampering of data to mislead either ALTO servers or clients.
        If some unauthorized and malicious party can retrieve data that is in transit or storage and tamper with it, clients would act on information that they assume is trustworthy but in fact has been modified.
        As such, clients could be redirected to wrong addresses, or receive incomplete or incorrect data that results in bad decision making.
        On the other hand, data tampering that occurs between data providers and the ALTO server would give it, a seemingly trustworthy party, untrustworthy data.
        This would result in the same issues that could arrive from spoofing threats.
        Tampering could also occur in input forms in the server-client or server-provider interface with potential to inject malicious code execution.
    \item Repudiation of being the source of some network information, whether it be by a third party that volunteered the data or the ALTO server itself, which would make it difficult to neutralize and attribute culpability to incorrect or malicious sources, jeopardizing the legitimacy of the provided network information .
    \item Information disclosure in the form of ALTO resources being made available to entities that were not contemplated to access it.
            These resources could give malicious parties insight on network topology status as well as the ability to derive clients' network usage patterns by observing what kinds of resources they attempted to retrieve in a given moment.
    \item Denial of service (DoS) of the ALTO server itself through query flooding beyond its capability, which would disable its ability to serve legitimate users.
        By proxy, service denial of external entities can also happen through the manipulation of ALTO resources themselves - leveraging the ALTO's potential to guide traffic, if a given resource is manipulated in such a way that unreasonable favors the preference of a specific subset of servers, these could be favourably picked by clients in a disproportionate matter, and highly affect these servers' availability.
    \item Elevation of privileges that lead to obtaining or modifying more information than initially permitted.
\end{itemize}

    Many of these threats are standard and could be solved with state of the art solutions which are well proven and tested, as indeed states \cite{alto-protocol}.
    However, threats of information disclosure - whilst they can be negated with in-transit encryption, what is done with this information the moment it reaches the client is hard to control - situations may arise when a client with proper resource permissions shares, intentionally or not, sensitive network information with one without, after interactions outside the ALTO architecture.
    Furthermore, many authenticated clients with different permissions could share information among themselves which they initially retrieved legitimately, to get a complete view of the network structure.
    Thus, individual clients could internally collaborate outside the system to bypass access control measures applied inside it.
    As such, it is firstly important for the ISP or third parties to carefully plan how what information they are comfortable with sharing, knowing that it may be susceptible to information disclosure outside the secure domain.
    Possible solutions to minimize these threats are as follows:

\begin{itemize}
    \item Reduce the granularity, and generally the details, of the provided data.
        Intuitively, the less granular and precise the shared information by the ALTO server is, the less valuable the resulting application guidance will be, and thus a balance would have to be found between layer cooperation and ISP privacy.
        One example is the usage of network groupings by PIDs instead of mapping information to concrete endpoints, working with network status around abstract entities.
        Another possible mean to reduce information granularity would be by utilizing ordinal cost values, which instead of specifying a concrete metric, e.g. bandwidth in bits per second or packet loss in percentage, the server would give a relative preference rating with lower costs meaning lower preference.
        In both examples, the granularity of network information transmitted to the client is several levels higher in abstraction than the actual physical layer, and this could reduce the flexibility of applications to optimize traffic.
        However, they can still provide acceptable flexibility without impacting ISP privacy, acting as a much needed compromise.

    \item Work only with a small set of trustworthy ALTO clients that are to act on behalf of a larger subset of less trustworthy clients.
        For example, network status resources could only be provided to authorized cooperation-oriented trackers in the BitTorrent protocol, which would in turn use this information to provide customized replies to clients without the need to change the base protocol.
        Similarly, information relevant for user-server assignment could only be provided to authenticated CDN control nodes, who'd share among them a private virtual domain to share information about user-server connectivity and server status that would otherwise be inappropriate for any other type of user.
        This is still, however, worthy of further threat analysis as restricted information could still leak outside of the system - beyond the means of spoofing discussed previously, seeing how a system behaves with ALTO guidance can give - albeit limited - insight into ISP bias.
        To see this, consider how a BitTorrent peer could continuously query a tracker with carefully crafted parameters - such as source address and candidate peers - and attempt to derive information from the resulting action, or similarly how a end-user could utilize similar parameter modification to observe the edge server selection mechanism in action.

    \item Utilize terms of agreements that are to be enforced on every querying client, stating that network status information does not get used beyond its original purpose, prohibiting sharing, and likewise that provided network information through third parties is not falsified.
        Although a potentially helpful mechanism to dissuade malicious users, it can be deemed impractical to apply, especially considering the scale at which this information could be shared.
        Thus, utilizing such means should be applied at a case-by-case situation and it should not replace ISP discretion and server resource maintenance to verify a given standard.
\end{itemize}

\subsubsection{Privacy}%

    Privacy concerns are also very prevalent in the ALTO system, being an ubiquitous talking point in most of the working group's problem statements and protocol specifications.
    When an ALTO client queries a server for one or more network status resources - in the attempt to optimize the application traffic it will generate in the near future - certain parameters can be passed to the server that can make the response be more personalized and contain more granular information.
    For example, a real-time P2P media-streaming application seeking ALTO guidance to help choose among a list of candidate streaming peers may wish to include in its query helpful parameters such as the peer list itself, the desired QoS metrics that should favor real-time media streaming, and the network position of the querying client itself.
    Indeed, these and more patterns will help increase the effectiveness of the ALTO server's guidance in helping the client application achieve its goal, but such happens at the expense of potentially allowing an ALTO server to infer on user pattern statistics.
    Even assuming that the previously discussed information disclosure threats are nonexistent in the ALTO system, privacy concerns can arrive from client applications because the resource queries they need to produce can contain information about what the client either will or wants to do.
    This is recognized by the ALTO working group as a possible concern \cite{alto-protocol} \cite{alto-problem-statement}.
    In response, they state that the clients should firstly be cognizant about the potential tracking risk that is associated with the usage of the system and, as an attempt to make tracking harder, they could disable HTTP cookies and/or opt for more vague query parameters, e.g. by randomizing some bits on endpoint addresses or simply using more broad addresses, whilst being aware that the helpfulness of query results may vary with increased parameter obfuscation.

    Privacy is likewise a concern towards the server side, as the ISP is providing sensitive network details to clients.
    Very much like client privacy, ISP-related privacy is also considered by the working group.
    Provider-Defined Identifiers (PIDs) were created as a means for ISPs to abstract network components as a collection of single network endpoints with similar properties, helping them not to disseminate network information that is too sensitive, and in turn also allows clients to make queries based on these identifiers and maintain a higher level of privacy.
    An ongoing proposal for protocol extension includes path vectors \cite{alto-path-vector(draft)}, that aim to represent information on the intermediary hops from a given source-destination pair, and each of these hops is represented as an Abstract Network Element (ANE) that, similar to PIDs, give ISPs the ability to under or over-abstract the topological representation that gets published to clients, giving more options to balance guidance usefulness with ISP privacy.
    Other solutions could also be considered depending on the needs of the clients and the direction of the project as a whole.
    For example, the servers themselves could operate on a secure communications channel and maintain a clear agreement on what can and cannot be made with the collected information.
    Alternatively, clients that wish not to impose much trust on the server's claims not to track them could make bulk queries (or use proxies to do so for them) and privately filter out the relevant information.

\subsubsection{Incentivisation}

    Incentivization relates to creating and divulging, to both layers, incentives to a fully cooperative layer relationship that is inherent in the oracle pattern adopted by the ALTO system.
    It is quite the challenge to fundamentally change how applications behave on the internet, as indeed it is to ask of ISPs to launch a view of their infrastructure to the outside world.
    \cite{dan-Commag10} notes incentivisation as one of the key challenges in overlay-underlay cooperation in regards to P2P applications, stating that incentive mechanisms need to exist to ensure that both layers both agree into and maintain a cooperative relationship.
    According to the ALTO problem statement \cite{alto-problem-statement}, the incentives for both parties to act on the system are the advantages that derive from using it - clients are to expect better application performance by leveraging ALTO guidance, and similarly ISPs should expect that their internal goals, such as an optimization of infrastructural utilization, can be met with the increased traffic engineering ability resulting from their oracle role.
    If the overlay consuming ALTO guidance has a manageable number of accountable entities - such as a single CDN or data center that the ISP agrees to partner with - it is realistic to maintain a cooperative agreement that can be solidified with feedback and service agreements.
    However, if the overlay utilizing the ALTO system makes it hard to pinpoint accountability, such as a large P2P application with many users, it will naturally be harder to ensure that the power dynamic between layers doesn't shift beyond an equilibrium.
    In these cases, policies should be created and enforced to give insurance to both parties that a cooperative relationship is maintained.

    The lack of cooperation could also occur by the ISPs, that could leverage their new-found application traffic engineering capabilities to steer traffic in a way that is advantageous to only them.
    Again, much like the lack of cooperation by clients, it is difficult to guarantee an equilibrium in the power dynamic between layers, but by guaranteeing improved QoS levels for applications that utilize ALTO cooperation, ISPs become responsible in guaranteeing that these improvements are met, fearing client abandonment otherwise.
    Giving freedom to both layers on how they act ensures that the system evolves to a common ground that benefits both sides, at least enough to justify them remaining there.

    Finally, if the application-ISP tussle becomes harsher, as it tends to, a cooperative system such as ALTO may become necessary - and thus beyond preferable - meaning that ISP may be forced to block or throttle traffic that it cannot route properly, as it historically happened.
    Thus, acting with ALTO goes beyond cooperation into symbiosis, meaning that both parties have to act cooperatively to maintain network sustainability.
    Regardless, the best approach seems to be that the system must in of itself be self-justifiable, meaning that the advantages that it brings should be enough to convince both parties to act on it.
    ISPs are nevertheless free to deploy their own incentive mechanisms to facilitate early application adoption, that could include monetary rewards or routing privileges.

\subsubsection{Network Neutrality}
    As stated by \cite{qos-framework}: "According to most network neutrality proponents, network neutrality rules are intended to preserve the Internet's ability to serve as an open, general-purpose infrastructure that provides value to society over time in various economic and non-economic ways. In particular, network neutrality rules aim to foster innovation in applications, protect users' ability to choose how they want to use the network, without interference from network providers, and preserve the Internet's ability to improve democratic discourse, facilitate political organization and action and to provide a decentralized environment for social, cultural and political interaction in which anyone can participate.".
    Network neutrality has been a popular point of discussion as society grows around the Internet, sparking debates around the world on what the best course of action should be - for example, regulations were introduced by the FCC in the United States \cite{fcc} to police network neutrality, and the European Union has a framework for net neutrality laid down by article 3 \cite{article-3}.
    However, potential violators of the spirit of a network neutrality exist, such as British Plusnet's usage of deep packet inspection (DPI) to implement limits and differential charges for different traffic \cite{arstechnica}, or Portuguese MEO's smartphone contracts which include zero rating programs for a given set of services \cite{meo-packages} that bundle applications such as Facebook or Spotify.
    Network neutrality advocates are concerned with guaranteeing that ISPs keep Internet communications free and do not discriminate based on the traffic's specifics, such as platforms, applications, or source and destination.
    On the other hand, opponents of net neutrality, among them ISPs, broadband and telecom companies, and hardware manufactures, argue against net neutrality - they claim that it would would reduce incentive to invest, as investments would be harder to insure without the ability to charge higher rates for better infrastructure capabilities.
    Zero rating programs, such as Wikipedia Zero, which provides Wikipedia with no charge to a select group of low income regions \todo{cite} , would also not be possible.
    Additionally, with net neutrality, the ISP's ability to route traffic could itself be at jeopardy - as \cite{jerzy} states in their solution to compromise net neutrality with QoS demands via service differentiation, the Internet is growing at an astonishing rate, as are the QoS demands of applications, and operating the infrastructure on a purely best-effort basis will not be sufficient without a constant provisioning of such infrastructure to keep up with demand, and this too may not be viable nor even possible.
    Thus, discriminating by traffic services may be needed to guarantee that, say, real-time medical information gets priority over real-time media streaming, which in turn gets priority over e-mail or file sharing.

    Considering that the ALTO system behaves in an oracle pattern of cooperation where a single entity - the ISP - is able to heavily influence the traffic patterns of the applications it aids, on the promise of a cooperative network underlay-overlay relationship, such system could violate the principles of net neutrality.
    In particular, this could happen if the oracle either blocks, or at the very least provides different guidance to different clients, depending on where the query originated from - e.g. what application, source address, or other defining characteristics.
    A possible consequence of such a system guiding the Internet could be that given applications can consistently have better QoS measurements not on the basis of the application's implementation, but on the ISP personal biases.
    Oracle systems such as ALTO do not seem to be analogous to other traffic engineering strategies, such as the usage of MPLS, DiffServ, nor to other means of ISP intervention on overlays, such as the deployment of data caches and redirector proxies - this is because the oracle system, in contrast to the previously mentioned strategies, is one of mutual voluntariness and cooperativeness between ISPs and applications.
    However, it could be argued that if the ALTO system offered guidance to applications in such a way that consistently resulted in better application performance, such applications would be pressured to use such guidance as a means to remain competitively viable, and the ISPs would then have a platform to influence a considerable amount of traffic to their will, being in a position to, depending on how they treat guidance requests, break network neutrality.
    This neutrality concern can be alleviated if application guidance operated on classes of traffic, e.g. real-time communication or file sharing, thus operating on traffic aggregates to insure QoS levels needed to given application types, but never discriminating beyond such given classes.

    As the protocol is defined \cite{alto-protocol}, the provided network status information is truthful and guidance is optional, and neutrality can then still remain outside of the system, since no routing measurements exist within it.
    If particular implementations of the ALTO system give guidance in such a way to guide traffic in a discriminate fashion, and if such guidance have advantages that much outweigh any alternative - thus rendering it beyond optional - a case can be made for how ALTO as a concept can break network neutrality - for all its advantages and disadvantages discussed bellow - as the ISP can utilize discriminatory behaviour to treat applications on their infrastructure differently.

\subsubsection{Multi-Domain orchestration}

    The Internet as we know it today spans the entire globe and is rather complex in nature.
    According to \cite{dan-Commag10}, the classic vision of the Internet consisting of a network of transit and stub ASs no longer seems accurate, as it now is much more complex - for starters, the role of network owner and service provider are separating, and Internet access is provided by numerous competing ISPs.
    As a demonstration of such complexity, Figure \ref{network-connectivity-globe} displays how the Internet is structured into many tiers of different service providers.
    It can thus be seen that the act of layer cooperation can get harder when the influence domain increases and potentially spans many different ISP regions which will inevitably act differently as they can have different technologies, biases, policies, and overall goals.
    These per-ISP biases can make it difficult to guarantee that traffic optimization spanning multiple administrative domains is actually useful and achieves the cooperative nature in mind - for example, an ISP may not be comfortable categorising end-point costs of a given metric, thus making path calculations that pass through that ISP domain not viable.
    Regardless, per-domain ISP guidance has nonetheless plenty of potential - for example, the ability to localize traffic, as entities outside of domain can be identified by ISP, and similarly per-domain optimization of resources can still be useful when such domain is large, and can be applied to high-volume operations such as those in a data center.
    The ALTO server within a given domain can also leverage probing measurements and feedback statistics to derive information in areas whose topological detail is unknown, giving a partial network view that contains topological insight and also information derivation that, whilst not being as good as a complete topological insight, may nonetheless power a cooperative effort within a given domain with good results.
    Some data may, whoever, be both not shared by an external domain nor derivable - endpoint property information, such as network connection types, or server footprints - e.g., available cpu, ram and storage - can only be retrieved by authoritative entities in a given domain and probing to them may be not available, thus considerably limiting the applicability of a single ALTO domain.

    \begin{figure}[ht]
    \centering
    \includegraphics[scale=0.30]{network-connectivity-globe.png}
    \caption{Conceptual representation of ISP diversity on the Internet}
    \label{fig:network-connectivity-globe}
    \end{figure}

    Even assuming that all ISPs are comfortable with sharing sufficient information, ambiguity may arise - considering a cost map with the generic "routingcost" cost metric, ISPs could internally calculate routing costs differently, and prioritizing different goals, e.g. reducing overall link usage versus reducing inter-AS traffic first and foremost.
    The base ALTO protocol specification states that each network region can provide its ALTO services, which in turn convey network information from their perspective.
    A network region consists of a given administrative domain, such as an AS, an ISP, or a given set of agreeing ISPs \cite{alto-protocol}, thus implying that if multiple ISPs share an ALTO server they must reach a consensus on what network status is available for query from the outside.
    Furthermost, the ALTO working group's deployment considerations \cite{alto-deployment-considerations} document states that an ALTO client can query a single server for one or many metrics, or he can additionally query multiple server instances on different networks \cite{alto-deployment-considerations}.
    It is explicitly stated in the document that each server could give guidance for only a given network partition, and such guidance may wildly differ between them due to the fact that different algorithms and objectives may have been applied.
    The document also states that, in regards to extending ALTO's reachability, three different strategies could be applied:

\begin{itemize}
    \item \textbf{Authoritative Servers}: A given set of servers can provide guidance for all kinds of destinations to all ALTO clients.
    \item \textbf{Cascaded Servers}: An ALTO server can possess an embedded ALTO client and query other ALTO servers if it cannot serve the original request, acting as a middleman between the client and the more apt server.
    \item \textbf{Inter-server Synchronization}: Different ALTO Servers communicate among themselves to expand the knowledge space.
\end{itemize}

    The last strategy is still being subject to development and standardization by the working group as part of a bigger attempt to link different network regions and technologies into a single, homogeneous abstraction of the Internet.
    Current efforts in multi domain orchestration and relevant use case examples are summarized in the ongoing work of \cite{ALTO-multi-domain-use-cases(draft)}.

\section{Summary}

\todo{Finish this}

\textbf{[Mention how in all overviewed cases (p2p, cdn, client-server) there is something to be gained on both sides by enabling layer cooperation. By better knowing how the network is structured, better application decisions can be made that minimize network resources, and this in turn is also befinitial for ISPs. All the seen cases suffer in how to better match given entities that wish to generate traffic, whether that be between peers, between edge clients and edge servers, or between clients and server replicas, respectively. ALTO's usefulness goes beyond this, and also provides a standardized means through which a centralized, well maintained  framework helps retrieve network status and ISP influence to increase application awareness. This helps, for example, not only to better select between redundant service entities, as seen before, but also to better decide when to perform a given network action in accordance to the current network status, how to select new and appropriate nodes to the application, how to extend the domain knowledge of the network, etc.]}

\chapter{System Architecture and Developed Mechanisms}

    As the main proposed goal of this work is the implementation of a system that complies with the ALTO working group's devised protocol, this chapter exhibits the planned software specifications needed to implement the system as a whole.

    Initial attention is given to the general architecture on the first section, with the goal of identifying key entities, their purpose, and how they interact among themselves.
    The following section will target the specification of ALTO resources, which can be considered the driving force behind the system, as they are what the client entities seek, and likewise what the ISPs wish to provide.
    The next section will focus on specifying the task of network status provision to an appropriate ALTO server, in such a way that a common interface exists among all entities that are able to increase the server's knowledge of the network's physical topology.
    Upon specified the way that network information is provided, the next section details how a given actor, such as an ISP administrator, can pre-process such information before it is forwarded to the server and available to clients.
    Finally, the task of multi-domain ALTO server synchronization and communication is specified in the form of required protocol extensions and needed mechanisms that allows the increase of a single ALTO server's knowledge space.

    \todo{add this : which includes the insertion of static ISP preferences or the abstraction of network entities as a means to dilute concrete topology details without forfeiting the usefulness the clients can retrieve from the processed resources}

\subsection{System Architecture}

    Figure \ref{fig:architecture-network} presents a high-level conceptual model of how the network information flows in a given ISP.
    Network data originates in the topology itself, and is gathered into a network information aggregator by the appropriate means - this aggregator defines an interface through which network data can be uploaded, and entities utilize it to provide the network data they have collected.
    These entities will use different means to gather different information, as the Internet consist of many different protocols and standards for network and resource information querying.
    For example, a node could deploy a daemon listening for Open Shortest Path First (OSPF) protocol packets to gather path cost information, and another using Simple Network Management Protocol (SNMP) to gather node property information.
    Obviously, since the interface simply defines how raw data must be formated to be accepted by the network information aggregator, the network administrator could use previously collected information that resides in a database and upload it as is.
    The network information aggregator serves as a hub for network administrators to process the raw network data that was collected by the previous tier, and transform it into ALTO resources ready to be accepted and distributed by the ALTO server.
    This task of network information processing is where ISP policies and preferences are injected via, for example, the abstraction of network entities with the aggregation of network addresses into PIDs, and the creation of cost maps which result from the transformation of network link information mixed with given ISP goals.
    If, say, the administrator wished to provide a cost map between network entities which aimed to reduce inter-network traffic, it would firstly aggregate endpoints into abstract entities with common properties, as an attempt not to share too much infrastructural information, and then use the provided network link information, attribute higher costs to undesired links, and transform it utilizing the Dijkstra's algorithm to create a shortest path map.
    Such map is then parsed as an ALTO resource and uploaded into the ALTO server with the access policies the administrator sees fit.

    As most software architectures, each new communication channel represents a possible attack vector and, attending to the critical security concerns posed in \ref{sssec:alto-security}, all pondered communication channels must be secure and reliable, as signified by the padlocks on the presented architecture.
    This implies that data communications within it must be resistant to being read or altered by outside parties, and the identity of the participating parties can be trusted and be made accountable.
    The identified communication channels must then have methods of maintaining data integrity in transit, user authentication and authorization, and communication confidentiality.

\begin{table}[!h]
\centering
\caption{Network node entities in the conceptual ALTO system representation}
\begin{tabular}{ | c | l |}
\hline
\textbf{Image} & \textbf{Description} \\ \hline
\raisebox{-.30\height}{\includegraphics[width=8mm]{img/circle-white}} & Network node \\ \hline
\raisebox{-.30\height}{\includegraphics[width=8mm]{img/circle-blue}} & Network node participating in a given overlay network \\
\hline
\end{tabular}
\end{table}

\begin{figure}[!h]
        \centering
        \includegraphics[scale=0.75]{img/architecture-network.png}
        \caption{Conceptual representation of the ALTO system of a given ISP}
        \label{fig:architecture-network}
\end{figure}

    More formally, Figure \ref{fig:macro-architecture} presents the proposed system architecture.
    One can identify the ALTO interface as a key component of the system, as it allows to bridge three different application layers - the ALTO resource consumer, the ALTO resource provider, and the network information aggregator, to be further specified in the following sections.

    The ALTO working group has extensively specified the ALTO protocol in regards to resource query, and the concrete implementation of this work will aim to comply to it.
    However, no resource provisioning protocol was, at time of writing, specified by the working group, nor was an interface been specified to allow network data to reach the ALTO server.
    It has been set as a work in progress, and the topic of network information supply sources was briefly discussed in \cite{alto-deployment-considerations}.
    The working group has grouped the tasks of raw network processing and supply into the role of the ALTO server.
    However, as can be seen in the proposed architecture, the roles were separated and an additional protocol proposed to bridge communication between them.
    This was made as an attempt to adhere to the philosophy of single responsibility, making the sole task of the ALTO server the management of ALTO resources.
    This aims to facilitate the independent development of the different roles, and make it easier to interchange implementations - this would make it particularly useful, for example, to deploy many ALTO servers in a cascade fashion whilst utilizing only a single network information aggregator.
    These are, however, only conceptually separated, and an implementation could, if it is more practical, merge the server and information provider roles into a single physical entity - which is similar to the architecture designed by the ALTO working group - if the conceptual roles and interfaces persist.

\begin{figure}[!h]
        \centering
        \includegraphics[scale=0.25]{architecture-macro.png}
        \caption{System architecture at a macro level}
        \label{fig:macro-architecture}
\end{figure}

\subsubsection{Role system}
\label{sssec:system-roles}

    As an access control measurement, the system will work with Role-based Access Control (RBAC) methods which, as the name states.
    Thus, every entity acting on the system must identify itself as a user, and a given user can have many roles.
    The ALTO resources - the main interest of this system as it is the data component requested by clients and managed by the servers - has associated to it an Access Control List (ACL), that maps, for a given set of roles and a special "catch all" role when none else apply, the list of user actions that are allowed to be performed to that resource.
    The available user actions are read, update and delete, meaning the ability to get, change the contents of, or remove the resource, respectively.
    This ACL is provided by the Network Information Aggregator whenever a new resource is inserted into the ALTO server.
    The ISP administrator that controls the aggregator not only then designs the resource itself - adding the information that it deems important whilst not too detailed to damage privacy - but also defining access control policies on that resource, which will be then applied by the server in future requests.

    Employing access control based on roles seems appropriate for this system since roles can be applied to, and thus group, many users, and indeed that seems to be applicable on real case deployments of the ALTO system - each given application can correspond to a group, and more intimate scenarios - such as a data center - can be grouped too.
    As a user can be conceded many roles, he can naturally act on the system with a role that fits the currently queried resource, if so applies.

    Access control mechanisms such as this will help mitigate security threats pertaining to the ALTO working group's architecture - e.g. having unwanted users reading or tampering with data - mechanism that is also based on roles can give better flexibility on access credence attribution, as well as better scalability since these credences can be applied of groups instead of managed on a user-by-user basis.
    However, for such mechanisms to be viable at all, authentication systems need to also be employed to help verify that the users are indeed who they are announcing to be, and nevertheless authentication seems to be a very important requisite of this system to mitigate spoofing security threats.

    Data breaches are not, however, totally mitigated with authenticity and access control mechanisms.
    After an entity gets a resource and acts outside the system, it becomes out of its control and these mechanisms cannot be employed.
    This means that there are no guarantees that the resources are shared outside of the system's domain and consequentially there are no security guarantees after that point.
    Because of this, privilege attribution by the ISP administrators not only give credence to do a certain action, but also imply that trust exists that these users will not be improper with the given resources and they won't thus share it with other users with improper credence.


\subsubsection{Resource Consumer}

    An ALTO resource consumer is materialized in the architecture in the form of an ALTO client, which can be any entity who is able to interface with an ALTO server to query for ALTO resources.
    Whilst the ALTO working group was initially devised to help increase traffic localization via the sharing of network information, it now has an increased scope where an ideal client is any application which generates network traffic and would be able to optimize it with aid from an oracle entity with privileged network information.
    Thus, an ALTO client is fit to be implemented in P2P applications, and could be embedded in a P2P client itself to help with picking neighbouring and content providing nodes, or on a tracker that would accomplish the same goal on behalf of the querying peer.
    Likewise, nodes which are unable to optimally select between other nodes, such as CDN edge nodes or content mirrors, could also benefit from oracle guidance, and thus qualify as appropriate ALTO clients.

    Figure \ref{fig:p2p-communication} exemplifies how a cooperative P2P application would, acting as an ALTO client, interact with the ALTO server to retrieve relevant network resources to aid their application choice of what candidate peer to consume a service from.
    Firstly, a network map is retrieved to help group endpoints into groupings, and afterwards a cost map is retrieved filtering only the querying peer as source, candidate peers as destinations, and the routing cost and bandwidth cost matrices.
    Acting on this information, the peer chooses the candidate that gives a good balance between ISP routing cost and path bandwidth, making a decision that should ideally benefit both them and the ISP that provided the information.

    Figure \ref{fig:p2p-tracker-communication} is similar to the previous example, meaning it relates to a P2P application, but this time the application-level traffic optimization is made in a way that is transparent to the P2P client.
    As a choice to purely localize traffic, as this alone can bring plenty of benefits to both layers, and as a means to minimize protocol modification, it is the tracker that serves as an ALTO client.
    Whenever a request is made by a P2P client to retrieve peers serving a given data chunk, the tracker first consults with the ALTO server and retrieves its network map that groups peers within administrative domains - inside the providing ISP's domain, thus the local network, and outside administrative domains.
    The tracker could use a very simple algorithm to filter out of its candidate pool candidate peers outside the local domain if results inside it exist.
    After packaging a reply to the P2P client, the protocol acts normally and traffic could be successfully localized.

    On the same vein, Figure \ref{fig:cdn-communication} exemplifies how this time a cdn controller would use the system to better help its decision in matching CDN clients to an edge server on their system.
    To do this, it retrieves a property map to query for server status information, and subsequently retrieves a cost map to query for path information between the CDN client and the candidate edge servers.
    Having all the relevant server status information, e.g. available processing and storage resources, and path properties, e.g. max bandwidth, latency, packet loss, the CDN controller is in a condition to more optimally redirect his client.

\subsubsection{Resource Provider}

    An ALTO resource provider is the ALTO server, an entity that possesses pre-processed network information in the form of ALTO resources.
    Its job is to store and manage such resources, and provide it to querying ALTO clients, and Integral to these responsibilities are also data validation and persistence.
    Conceptually, the ALTO server is seen as a single entity, but considering the sensible information that could be stored within it and the influence it has on shaping network traffic, it would not be uncommon for an ALTO server to have a knowledge domain correspondent to the ISP that owns it.
    Physically, though, the resource provider layer could consist of many interlinked ALTO providers with an increased coverage area of network knowledge.
    Means through which this could occur are further specified in section \ref{ssec:multi-alto}.

\subsubsection{Network Intelligence aggregation}

    The network intelligence aggregation layer is the layer that enables the translation of raw topological information - such as the physical attributes of network devices and connections - into processed, query-eligible network knowledge.
    To do so, a very important entity, perhaps the heart of the system as a whole, is the network state provider, which is the supply of network information that is injected, through a network state provisioning protocol, into a network information aggregator.
    This latter entity is then responsible for providing the ALTO resource provider layer with valid information after the raw topological data has been processed - this includes the calculation of optimal paths, the abstraction of network entities, or the injection of static ISP preferences.
    This pre-processing stage requires input from an ISP administrator, responsible for acting on the best behavior or the ISP from which the raw topological data originates - by interacting with the network information aggregator the administrator acts on this network information hub to retrieve from the database a history of retrieved network information, and afterwards manipulate this information to create ALTO resources to its liking - this is where data is transformed utilizing the algorithms the admin deems fitting, and transforms the raw data to be publishing ready, meaning that it contains an acceptable amount of abstraction not to compromise topological privacy.
    Finally, the admin defines important meta data that identifies the resource, and defines the access control list to be enforced by the ALTO server.

\subsubsection{Resources}
\label{ssec:alto-resources}

    ALTO resources are pieces of network information which are provided by an ALTO server and consumed by ALTO clients that ideally would use such information to aid their application-level traffic decisions.
    All ALTO resources can be separated into the following:

\begin{itemize}
        \item \textbf{Meta information}: data which regards to the resource's profile, that enable the client's ability to interpret and cross-reference the network data within.
            Following suit to the defined protocol, meta information contains the resource's name, if applicable, version, resource dependencies and cost details - enclosed cost modes, metrics, and descriptions.
            Finally, belonging to the meta section of the resource's information is the resource's ACL which, to a given set of roles, specifies what action those roles can perform on the resource.

        \item \textbf{Network status information}: data structures that give a characterization of the ALTO Server's vision of a network. Concretely, these can map network properties to a node (such as the connection types of their interfaces, or their geographical location), they can aggregate many network addresses to a single identifier, or they can map properties to a node link or end-to-end path (such as link or cumulative routing costs).
\end{itemize}{}

    Meta information can be seen as a resource's header, containing data that regards to the network status and helps better handle it.
    Following the defined protocol \cite{alto-protocol}, this field includes the resource's name for all resources which is needed for identification, and all other fields are dependant on the type of resource: at this version of the protocol, only network maps are version-able, allowing ISPs to reference different versions of a network map as this is updated; cost information is, naturally, only applicable to cost maps, and gives insight on how the numeric costs are to be interpreted, i.e. what their mode and metrics are, and what description it has.
    Finally, extending to the protocol is the addition of an ACL as a solution to access control needs.
    An ACL is defined as a matrix, with each entry defining a user role and actions - discussed in \ref{sssec:system-roles} - as a restriction on what a given user was given clearance to do.

    The network status information of a network map groups endpoint addresses into a single PID as a text literal.
    Akin to the ALTO protocol \cite{alto-protocol}, accepted endpoint address protocols include IPv4 and IPv6, utilizing a 32 bit long bitmask to identify a subnetwork.
    Similarly, support for aggregation of MAC addresses was added, with a 48 bit long bitmask to identify address ranges, similar to the IP variant.
    Additionally, generic overlay IDs can be added with the key priv:X - meaning private scheme - where X is the qualified name - this naming scheme was adapted from the endpoint property map's specification for additional property names, for semantic consistency.
    As endpoint addresses utilizing this scheme can be anything, the their interpretation is also left to the client - for example, if a server defines that PID aggregator of priv:my-overlay can use regular expression to specify address ranges, a pre-agreement must exist with a client.
    Of course, if a given addressing scheme besides the previously mentioned ones becomes popular, it could afterwards become part of the specification, but the existence of a private addressing scheme with liberal usage gives liberties outside of the protocol, if so are needed.
    A valid network map must unambiguously map every address in the domain range to a single PID, and whenever multiple matches occur wins the longest prefix match.
    As the custom addressing schemes let the network mad be interpreted in an undefined way by the protocol, the server cannot properly assert to the matching validity, and thus default protocol addressing schemes for network maps should be preferred, as semantic validity in private addressing schemes is not checked.
    Table \ref{table:networkmap-example} provides an example network status component of a network map within the topology in Figure \ref{fig:example-topology-boundary}.
    Three PIDs are given, each taking portion of an IPv4, IPv6, MAC, and custom overlay address range.
    The private address scheme groups users in regards to their private overlay ID, and it can be seen that nodes with ID 1, 3, and 4 are grouped to a single PID, which we can see belong inside the ISP domain.
    Lastly, nodes 2 and 5 are given different PIDs as they reside outside the ISP domain but are reachable through different peering points.
    The ISP could thus leverage the network map to define domains of locality, as well as defining two different external groups, as these two associated nodes are reachable through different peering parameters.

\begin{table}[]
\caption{Example network status information of a network map}
\begin{tabular}{llllc}
PID  & IPv4          & IPv6                  & MAC                  & \multicolumn{1}{l}{priv:my-overlay} \\
PID1 & 10.20.3.0/24  & \multicolumn{1}{c}{-} & D0-9F-BF-2A-00-00/32 & [1, 3, 4]                           \\
PID2 & 10.20.3.10/25 & \multicolumn{1}{c}{}  & D0-9F-BF-2A-FE-00/40 & 2                                   \\
PID3 & 0.0.0.0/0     & ::/0                  & 00-00-00-00-00-00/48 & 5
\end{tabular}
\label{table:networkmap-example}
\end{table}

    A cost map consists of a list of cost map matrices, with each matrix setting pairwise values between an origin entity and a destination entity.
    If it is a standard cost map, these entities are represented by PIDs that can be cross-referenced from a network map which this resource depends on, and if it is an endpoint cost map, these entities are endpoint addresses which, similar to network maps, include IPv4, IPv6, MAC and private endpoint types.
    A matrix must specify the type of cost represented with both their cost type and cost mode, with available options being the ones specified in \cite{alto-cost-metrics(draft)}.
    Optionally, a cost matrix can specify calendar information about that matrix - similar to the current work in \cite{alto-calendar-cost-map(draft)} -  which signifies that besides having single-value costs - which are obligatory for any cost matrix - it also contains a time-sensitive list of costs that must be interpreted according to the calendar information provided.
    Table \ref{table:costmap-example} provides an example of cost matrices within a single cost map within the topology in Figure \ref{fig:example-topology-boundary}.
    As can be seen, a generic "routingcost" cost matrix - whose value increases with the associated costs of transferring data through that path, and derived as the ISP best sees fit - lets the ISP communicate routing preference to the client - costs within the ISP domain, i.e., from and to PID1, incur minimal cost, whereas paths that originate locally and target PID2 or PID3 - both utilizing peering links - are less preferable, with the former being two times less preferable than the former.
    A "packetloss" cost matrix is also provided, with the ISP applying preceding probing measurements between endpoints and averaging the values within PID groupings.
    In this case, locality is correlated with more reliable communications, and in situations where outside service is required, the cost matrix provides insightful information to help select the better peering connections.
    Finally, a "bandwidth" cost matrix can be seen, with the ISP applying probing measurements, topological insight, as well as collected feedback of previous application connections that occurred between endpoints to deduce theoretical available bandwidth between target points.
    Additionally, the inclusion of cost calendar capabilities to the cost matrix enables users to get a chronological view of bandwidth availability as rush hour arrives, with the single value cost being updated to the present time if a decision needs to be made only considering the current time.
    Note that, if more applicable, the ISP could've chosen to - in alternative or addition to - provide an endpoint cost map, which instead of having costs between groupings specifies endpoints.
    In this particular case, an ISP could've preferred to use groupings as a means to increase topological abstraction and concluded that the groupings can achieve functionally similar results.


\todo{remove link 2-5}
\begin{figure}[!h]
        \centering
        \includegraphics[scale=0.75]{img/topology.png}
        \caption{Example network topology with ISP boundary}
        \label{fig:example-topology-boundary}
\end{figure}

    The network status information of an endpoint property map stores the property information of a given endpoint.
    The ALTO working group's protocol specification \cite{alto-protocol} does not directly specify what kind of properties are pondered for this map.
    Following the same design pattern used for the other specified resources, the endpoint property map will have a set of defined properties with associated semantics, and all other properties can be added with the "priv" prefix to designate private properties outside of the considered domain, and thus all semantics and validation rules don't apply.
    Much like the other resources, an endpoint can be identified by an IPv4, IPv6, MAC or private overlay address, and the pondered properties are PID value, geographical coordinates, connection type (fiber, adsl, etc), server footprint information (total ram, storage, and processing power), and server status information (what part of the footprint information is currently available).
    In practice, a given property could be promoted from a private type to one pondered in the protocol and have a resulting official semantic and validation rules.

    Finally, as a means to facilitate resource divulgence from servers to clients, there is also included the specification of an Information Resource Directory (IRD), that is based from the ALTO working group's protocol specification.
    An IRD can also be thought of as a resource, but instead of sharing network information it serves as an index of the available resources that a given server provides.
    Each server must provide a single IRD, and it contains as many resource attributes as the amount of attributes it provides.
    Each resource attribute must contain the resource's ID and its HTTP media type and, if applicable, their capabilities, accepted input media types, and dependent resources.
    The capabilities identifies, if existing, the cost and property types that are used - being indexed by their unique name, this allows for these to be cross-referenced on further protocol exchanges without need to repeat information.
    Additionally, the resource's capabilities also serve to indicate what resource functionality extensions are enabled - currently applicable for cost maps only, it serves to signal if the cost map has calendared costs, if it accepts input filters, or if multiple cost matrices can be requested at once.
    Two additions are made to the working group's specification - firstly, a description field, which for each resource attribute gives a brief description of what it is about, as it could facilitate resource selection since such a description could go into detail about appropriate usage guidelines of that resource and suggested use cases; finally, the resource's ACL, letting a user know beforehand what credences he has in relation to the server's resources - being a crucial part of metadata of the resource, it seems fitting to go into the IRD, and has the added benefit of giving the user this crucial piece of information without him having to make resource requests just to, by server reaction, figuring out what he can and cannot do.

    Further formal specification is not made as it has been extensively done in the ALTO protocol \cite{alto-protocol}, and the proposed system is compliant to it whilst extending upon the design.

\subsection{Network status provision}

    Before ALTO resources are provided into the ALTO server by the Network Information Aggregator, the latter needs himself to be provided with raw network status information.
    The ALTO working group has discussed possible sources of raw topological information, including protocols like IGP, BGP, SNMP, or NETCONF, or databases like the Traffic Engineering Database (TED) or Label Switched Path Database (LSPD) \cite{alto-deployment-considerations}.
    A protocol needs to exist to interface between the entities that collect and provide the raw topological data, and the Network Information Aggregator that processes it and provides it to the ALTO server.


\todo{communication diagrams}
\todo{resources - source name, origin measurement time, description, endpoint/pair of endpoints, measurement}
\todo{interface}

\subsection{Network intelligence preprocessing}

\todo{communication diagram}

\subsection{Server discovery}

\subsection{Resource Filtering}

\subsection{Inter-server communication}

    \todo{add img/topolgy.png but more complex to showcase multi domain problem}

    A glaring gap in the working group's base ALTO protocol is its single administrative applicability domain.
    Meaning, an ALTO server is managed by a single administrative entity - likely an ISP - and its knowledge domain is limited by the network topology details that the entity wants to share.
    In the attempt to fix the server's inability to provide network status information outside its domain, this section overviews mechanisms that enable inter-server communication as a means to expand the capabilities of each domain.

    Firstly, consider how efforts for full resource synchronization could be taken.
    These would be similar to data synchronization mechanisms employed by popular databases to ensure consistency across several server replicas, and could increase availability as well as the serviceability of content nearby clients.
    However, it does not seem to fit this use case - for starters, if all data were to exist redundantly on all servers, that would defeat the purpose of having many administrative domains and thus a single server architecture would suffice; secondly, the architecture is inherently designed to work within a trust domain of selected clients, and because of it the servers may not even be comfortable with sharing all of its information within other domains to begin with, limiting replication strategies; thirdly, accounting for the amount of users acting on the ALTO system, better scalability could be achieved with a distributed solution that limits information within set boundaries.
    Accounting for these reasons, an inter-server synchronization protocol was designed for servers to negotiate information exchange among themselves.

\todo{communication diagrams}

\label{ssec:multi-alto}

\chapter{Implementation}

    Following the specification, the aim of this chapter is to overview aspects of the implementation stage of the proposed ALTO system.
    Firstly, attention is given to the chosen technologies that were leveraged to get the system from its specification stage into a working product - this includes tools and frameworks in the development and deployment phases, and whose choice greatly delimits the system's properties.
    Secondly, the server is put into spotlight by detailing how it is structured and how it behaves, taking special concern in how object oriented programming was leveraged to maximize modularity and reasoning of the system to facilitate future alterations and extensions.
    Special attention is given to the system's security, where an overview is made what concrete steps were taken to nullify the previously detected potential threats that put into question the viability of the system.

\section{Technologies used}

    Starting the implementation stage of every project, attention must be given into what tools are selected to make it come to fruition.
    These can greatly impact the success of the developed software, and has concrete consequence in its maintenance and future extension.

    The specified system architecture is composed by key entities whose interactions among themselves is clearly defined by interfaces.
    More so, considering the example deployment scenarios, its evident that each entity resides in different topological regions throughout the network, with then the ALTO server, clients, and status providers being scattered through an ISP domain.
    Each entity can then be thought of as a self contained system in of itself who must abide by the proposed interfaces to properly work on the system.
    A logical conclusion to this is that each entity implementation is independent from the next, needing to only assure a common communication channel that all entities within it can properly understand.
    This gives great flexibility in the system implementation as a whole, because different tools can be leveraged to different entities if needed, and such entities that be worked on independently from the rest without impacting the function of the group.

    Regarding the ALTO server, first attention is given to the ALTO resources that must be provided by it.
    They all share some common properties - such as resource id, ACLs, owner, etc. - and functionality - such as the ability to be read and updated, or their permissions modified.
    This similarity is further intensified within groups of resource types, more specifically cost maps - consider how the only concrete difference between and endpoint and a PID cost map is the type of entity the costs refer to, which are endpoint and PID addresses, respectively.
    The sequence of steps that must be taken from the initial point where a client requests a resource up until that resource is provided can be abstracted as the sequential interaction between concrete modules that have a given, self contained, purpose, and communicate with a common interface - an architectural pattern that is a micro version of the one existing in the system, that shares all its properties that were discussed previously.
    Some of these modules would include client request monitoring, its parsing, its validation, database retrieval, database storage, and serialization.
    With all this in mind, an object-oriented programming seems like a proper fit to the complexity pertaining to the ALTO server, especially considering how future extensions would be likely, as very much will be in case of the ALTO working group.
    Working with objects as the base programming entity, many of the highlighted similarities between resources will become easy to be put into evidence, and module encapsulation and interfacing are natural and thus arguably easier to develop and maintain.
    The choice of a language that follows an object-oriented paradigm

    \todo{dynamic vs static}
    \todo{java as a mature programming language that is stable and has many mature libraries}
    \todo{intellij idea}
    \todo{rest api}
    \todo{spring framework as versatile, efficient, http controller support for rest api, security support, dependency injection}
    \todo{unit testing and integration testing}
    \todo{database discussion}
    \todo{mongodb as a scalar choice}
    \todo{database schema}

\section{Server architecture}

    The macro-level architectural diagram specified that the server's role is to serve incoming requests by clients and providers, and to interface with a database to persist resource storage.
    The server will implement a Representational state transfer (REST) interface leveraging the Hypertext Transfer Protocol (HTTP) as this pair is widely accepted and ubiquitous on the Internet, but also due to the fact that its resource-oriented interface standards fit nicely into the specified interface for ALTO server interactions, which too revolve around resource manipulation, and by adhering to proper REST designs good scalability can be achieved due to its stateless nature and potential for resource caching.
    The choice of HTTP as an application protocol fits nicely into a philosophy of leveraging existing and well proven protocols and technologies to increase the project's success, and indeed so was done to integrate authentication and encryption mechanisms.

    To accomplish this interface implementation, the internal server architecture will be, at a macro level, as shown in Figure \cite{fig:alto-server-architecture-macro}.
    This three-layered architecture consists firstly by a controller layer that intercepts communication requests, which after parsed and validated are redirected to the business layer, which in turn employs business logic to help satisfy the controller's requests, which may require a subsequent layer descent into the data layer via database queries.

    Figure \cite{fig:controller-unversioned-architecture} displays a class diagram focusing on controller classes that deal with resources that are not susceptible to version control - this includes every resource except the network map.
    As can be seen, all concrete controllers - such as an endpoint property map controller - are extensions to a generic controller class that is parametrized by its Data Transfer Object (DTO) , filter DTO, and service instances.
    This design choice was made because all controller logic that regards to resources without version control are the same, and by creating generic classes with type parametrization code reutilization is increased.
    The parametrization required by the controller is required to pass a concrete instance of the unversioned resource service generic class, which in of itself requires parametrization in resource DTO and resource filter DTO.
    By reflecting on common controller and service behaviour between all version control lacking resources, the conclusion was that working around generic classes maximizes reutilization, facilitates reasoning and decreases potential error.
    To help better visualize the result, refer to how the generic controller is implemented in \cite{lst:generic-controller}.
    To retrieve a resource, simply call the service class with or without the proper filter, depending on which method was triggered, by calling the appropriate methods that must implement the resource service interface.
    For example, an endpoint property map controller implementation simply extends the generic controller by providing the concrete DTO, filter DTO, and service implementations, as seen in \cite{lst:endpoint-property-map-controller}.

    \begin{lstlisting}[label={lst:generic-controller}]
    public class ParametrizedUnversionedResourceController<ResourceDTOType,
                                                           ResourceFilterDTOType,
                                                           ResourceServiceType extends ALTOUnversionedResourceService<ResourceDTOType, ResourceFilterDTOType>> {

    private final ResourceServiceType resourceService;

    @Autowired
    public ParametrizedUnversionedResourceController(ResourceServiceType resourceService) {
        this.resourceService = resourceService;
    }

    @PreAuthorize("@ResourceAuthorizationService.hasPermission(authentication, #resourceId, T(com.example.restservice.dto.security.PermissionDTO).READ)")
    @RequestMapping(method = RequestMethod.GET, value = "{id}")
    public ResourceDTOType getResource(@PathVariable(value = "id") String resourceId) {
        return resourceService.getResource(resourceId);
    }

    @PreAuthorize("@ResourceAuthorizationService.hasPermission(authentication, #resourceId, T(com.example.restservice.dto.security.PermissionDTO).READ)")
    @RequestMapping(method = RequestMethod.POST, value = "{id}")
    public ResourceDTOType getCostMapWithFilter(@PathVariable(value = "id") String resourceId,
                                                @Valid @RequestBody ResourceFilterDTOType costMapFilterDTO) {
        return resourceService.getResource(resourceId, costMapFilterDTO);
    }
}

    \end{lstlisting}

    \begin{lstlisting}[label={lst:endpoint-property-map-controller}]
    @RestController
    @RequestMapping("endpointprops")
    public class EndpointPropertyMapController
        extends ParametrizedUnversionedResourceController<EndpointPropertyMapDTO, EndpointPropertyMapFilterDTO, EndpointPropertyMapService> {

        @Autowired
        public EndpointPropertyMapController(EndpointPropertyMapService resourceService) {
            super(resourceService);
        }
    }

    \end{lstlisting}

    The same reasoning was used to implement the network map controller, but since this is the only resource that accepts versioning, the correspondent generic controller behavior does not contain similar behavior to the one above, and thus another was created, as seen in \cite{lst:versioned-controller}.
    The service layer implementation must now let the controller retrieve either a specific version of a resource, or the most recent one if no version is specified by the client.


    \begin{lstlisting}
    public class ParametrizedVersionedResourceController<ResourceDTOType,
                                                     ResourceFilterDTOType,
                                                     ResourceServiceType extends ALTOVersionedResourceService<ResourceDTOType, ResourceFilterDTOType>> {

    private final ResourceServiceType resourceService;

    @Autowired
    public ParametrizedVersionedResourceController(ResourceServiceType resourceService) {
        this.resourceService = resourceService;
    }

    @RequestMapping(method = RequestMethod.GET, value = "{id}")
    @PreAuthorize("@ResourceAuthorizationService.hasPermission(authentication, #resourceId, T(com.example.restservice.dto.security.PermissionDTO).READ)")
    public ResourceDTOType getVersionedResource(@PathVariable(value = "id") String resourceId,
                                                @RequestParam(value = "version", required = false) String resourceVersion) {
        return resourceVersion != null
                ? resourceService.getResourceVersion(resourceId, resourceVersion)
                : resourceService.getLatestResourceVersion(resourceId);
    }

    @PreAuthorize("@ResourceAuthorizationService.hasPermission(authentication, #resourceId, T(com.example.restservice.dto.security.PermissionDTO).READ)")
    @RequestMapping(method = RequestMethod.POST, value = "{id}")
    public ResourceDTOType getVersionedResourceWithFilter(@PathVariable(value = "id") String resourceId,
                                                          @RequestParam(value="version", required = false) String resourceVersion,
                                                          @Valid @RequestBody ResourceFilterDTOType resourceFilter) {
        return resourceVersion != null
                ? resourceService.getResourceVersion(resourceId, resourceVersion, resourceFilter)
                : resourceService.getLatestResourceVersion(resourceId, resourceFilter);
        }
    }

    \end{lstlisting}


    \todo{class diagram}
    \todo{emphasis on single responsibility}
    \todo{discuss packages}
    \todo{code reutilization}
    \todo{interface oriented with code injection}
    \todo{code example}
    \todo{MVC architecture}

\section{Network information aggregator}

\section{Security}
    \todo{concrete protocols used - https, http digest declined over other one}


\chapter{Experiments}

    The purpose of this chapter is to overview the experiments phase of the project, which consists of the work done to deploy and measure how the system performs in a simulated scenario.
    Whereas the developed unit tests in the implementation stage aim to verify the correct functionality of separate units of code pertaining to the system, the execution of the entire system as a whole to serve a set of hypothetical use cases can help achieve a better grasp on how correct system functionality among all the tested units.
    Adjacent to the goal of testing the system in deployed scenarios, the experiments phase also aims to embed in the simulated environment a list of application scenarios that could leverage the ALTO system to its advantage, and subsequently observe and measure if and how the ALTO server can help the client in the form of provided resources that guide the client in taking application decisions that constitute a win-win scenario between the overlay and underlay.
    As comparison, other known application-network interaction strategies will also be observed and their results measured as a means to compare the impact that these strategies have on both layers.
    Findings on the state of the art of existing interactions and the proposal of the ALTO protocol made on section \ref{sec:state-of-art}, together with the specified system extension on \ref{sec:specification} leads one to believe that a theoretical mutually beneficial scenario exists in an ALTO approach that cannot exist with more asymmetrical means of interaction.
    This chapter, however, puts those theoretical scenarios into a practical environment that could be replicated by those reading this work, and exposing the created scenarios and collected data can corroborate the theoretical conclusions, as well as leaving an opportunity for future discussion on how the system behaved, including its performance, its success in aiding clients, other existing client options that could be a better route, system shortcomings, etc.
    This discussion benefits the ALTO project and can give more maturity to the system as it was put through a simulated deployment against other common strategies.

    The first section displays the chosen technologies for tasks pertaining to the experiments.
    The next section focuses on the required steps taken to setup the testing environment.
    This includes the design and deployment of a network topology in a simulated environment, the creation of mock applications to serve as clients for the system, and the design and deployment of application and network status measurement tools.
    The following section individually overviews the devised scenarios to test in the simulation, and with it experiment specifics such as the initial problem, what strategies will be tested to solve it, how many runs will be made per strategy, and what metrics will be measured.
    Finished the experiments, the following section will display the obtained results that were collected in the simulated environment, and the section after that will discuss these results and how they fare with the theoretical findings.

\section{technologies used}

    \todo{core}
    \todo{docker}
    \todo{python - client apps}
    \todo{python - graphs}
    \todo{vnstat?}

\section{setup}

    \todo{core topology - image, emphasis on diversity of properties and connections, as well as heterogeneity and interaction of multiple domains}
    \todo{mock file sharing app with tracker}

    \todo{client can be embedded on any existing protocol}
    \todo{p2p mock application and tracker w/ embedded ALTO client}
    \todo{media mock application w/ embedded ALTO client}
    \todo{data center mock application w/ embedded ALTO client}

\section{scenarios}
\section{Results}
\section{Discussion}


\input{chapters/conclusion

\bookmarksetup{startatroot} % Ends last part.
\addtocontents{toc}{\bigskip} % Making the table of contents look good.

\bibliography{dissertation}

\printindex

\umappendix{Appendix}

\chapter{Support material}

%\chapter{Details of results}

%\chapter{Listings}

%\chapter{Tooling}

%Anyone using \Latex\ should consider having a look at \TUG,
	%the \tug{\TeX\ Users Group}


% Back Cover -------------------------------------------
\umbackcover{
	NB: place here information about funding, FCT project, etc in which the work is framed. Leave empty otherwise.
	}

\end{document}
