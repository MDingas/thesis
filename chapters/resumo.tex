\chapter{Resumo}

    Com o uso cada vez mais acrescido da Internet global que acompanha o início da nova década, a necessidade para otimizar esta rede global de computadores passa a ser uma grande prioridade na esfera tecnológica, que vê o seu número de utilizadores a aumentar, assim como a exigência, por parte das aplicações, de novos padrões de qualidade de serviço, como se vê no streaming de conteúdo multimédia em tempo real ou em interações de realidade virtual.

    Face ao aumento de tráfego e a padrões de exigência aplicacionais mais restritos, uma melhor compreensão é necessária de como os provedores de serviços Internet (ISPs) devem gerir os seus recursos.
    Numa tentativa por otimizar a Internet, um ponto fulcral passa no facto de como as aplicações utilizam a infraestrutura tecnológica da rede sobre a qual residem para atingir os seus objetivos.
    As aplicações Peer-to-Peer (P2P), por exemplo, são tradicionalmente desfavorecidas pelos ISPs devido à sua imprevisibilidade e incapacidade em serem controladas, bem como uma utilização pouco eficiente de recursos de rede, que acabam por resultar em maiores custos de operação para ISPs e graus de QoS impossíveis de atingir.
    Um problema aparente que pode ser apontado por estas e outras aplicações é a falta de consideração que estas têm pelas preferências dos ISPs durante a sua operação, e tal pode ser exemplificado no caso das aplicações P2P na falta de esforço em obter proximidade topológica com os vizinhos criados na rede overlay, que caso existisse seria preferível por administradores de rede e teria também potencial para melhorar o desempenho aplicacional.
    Todavia, uma tentativa de melhor esforço por parte das aplicações em operar segundo preferência dos ISPs não será bem-sucedida se tais preferências não são claramente comunicadas, e como tal existe tremendo potencial numa framework que possibilite uma via de comunicação entre estas duas entidades.

    O propósito primordial desta tese é o da implementação e extensão do projecto desenvolvido pelo grupo de trabalho Application-Layer Traffic Optimization (ALTO), que desenvolveu um protocolo de requisição de recursos no qual entidades autoritativas e fidedignas disponibilizam auxílio a aplicações na forma de informação de estado de rede e preferências administrativas, com o intuito de obter cooperação entre as camadas underlay e overlay durante normal operação aplicacional, e consequentemente obter uma melhor harmonia na Internet através de uma otimizada utilização de recursos e uma minimização de custos.
