\chapter{Resumo}

    Com o acrescido uso da Internet que acompanha o início da nova década, a necessidade de otimizar esta rede global de computadores passa a ser uma grande prioridade na esfera tecnológica que vê o seu número de utilizadores a aumentar, assim como a exigência, por parte das aplicações, de novos padrões de Qualidade de Serviço (QoS), como visto em domínios de transmissão de conteúdo multimédia em tempo real e em experiências de realidade virtual.

    Face ao aumento de tráfego e aos padrões de exigência aplicacional mais restritos, é necessário melhor compreender como os fornecedores de serviços Internet (ISPs) devem gerir os seus recursos.
    Um ponto fulcral é como as aplicações utilizam os recursos da rede sobre a qual residem.
    Muitas destas aplicações não têm consideração por preferências dos ISPs, como exemplificado pela sua falta de esforço em localizar tráfego, e o contrário seria preferível por administradores de rede, bem como teria potencial para melhorar o desempenho aplicacional.
    Uma tentativa de melhor esforço, por parte das aplicações, não será bem-sucedida se as preferências administrativas não forem claramente comunicadas.
    Portanto, um sistema que sirva de ponte de comunicação entre camadas tem potencial para fornecer um cenário mutuamente benéfico.

    O foco principal desta tese é o grupo de trabalho Application-Layer Traffic Optimization (ALTO), que foi formado pelo Internet Engineering Task Force (IETF) para explorar estandardizações de recolha de informação da rede. 
    Este grupo especificou um protocolo de recolha onde entidades autoritárias disponibilizam recursos contendo informação de estado de rede, bem como preferências administrativas. 
    A partilha de conhecimento infraestrutural é feita para possibilitar, durante operação aplicacional, um ambiente cooperativo entre redes overlay e underlay, para possibilitar uma mais eficiente utilização de recursos e a consequente minimização de custos operacionais.

    Este trabalho pretende dar uma visão da histórica disputa entre aplicações e ISPs, assim como apresentar o projeto do grupo de trabalho ALTO como solução, implementar e melhorar sobre as suas ideias, e finalmente verificar a eficiência do sistema numa simulação, quando comparado com alternativas clássicas.

    \medskip

    \textbf{Palavras-Chave:} Application-Layer Traffic Optimization, Content Distribution networks, Engenharia de Tráfego, Otimização de rede, Peer-to-peer
