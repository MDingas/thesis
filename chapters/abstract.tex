\chapter*{Abstract}

    With the ever-increasing global Internet usage that is following the start of the new decade with no intentions of stopping, the need to optimize this world-scale network of computers becomes a big priority in the technological world, as user demands increase and so do the Quality of Service (QoS) demands for applications in domains such as media streaming or virtual reality.
    In the topic of optimizing the Internet, one main concern regards itself to traffic that is generated at the application level.
    Peer-to-peer (P2P) applications are a popular example of application types which are classically unfavored by ISPs due to their unpredictable and uncontrollable nature that utilizes network resources less efficiently, resulting in higher costs and non-achievable standards of QoS.
    One main issue with these applications is the fact that these act with little consideration of the underlying network infrastructure on which they operate, with a popular example of this being the lack of care by P2P applications on the network proximity to the peers among whom they select for neighbouring relationships.

    This thesis aims to implement and extend upon the ideas of the Application-Layer Traffic Optimization (ALTO) working group, which devised a request-response protocol where privileged and trustworthy entities provide to applications information that regard to the underlying network structure where such applications run on, with the intent of achieving layer cooperation during normal application operations as a means to achieve better Internet harmony so it can better be prepared to the needs of the present and the future whilst minimizing operational costs.

