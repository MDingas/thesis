\chapter{Abstract}

    With the ever-increasing Internet usage that is following the start of the new decade, the need to optimize this world-scale network of computers becomes a big priority in the technological sphere that has the number of users increasing, as are the \gls{qos} demands by applications in domains such as media streaming or virtual reality.

    In the face of rising traffic and stricter application demands, a better understanding of how \glspl{isp} should manage their assets is needed.
    As an effort to optimize the Internet, one important concern is how applications utilize the underlying network infrastructure over which they reside.
    \gls{p2p} applications, for example, have been classically unfavored by \glspl{isp} due to their unpredictable and uncontrollable nature, as well as an inefficient utilization of network resources, increasing \gls{isp} costs of operation and limiting the \gls{qos} levels they can ensure.
    An evident issue is that most of these applications act with little regard for \gls{isp} preferences, as can be evidenced by their lack of care in achieving network proximity among neighboring peers, a feature that would be preferable by network administrators and that could also improve application performance.
    However, even a best-effort attempt by applications to cooperate will hardly succeed if \gls{isp} policies aren't clearly communicated to them.
    A system to bridge layer interests has thus much potential in helping achieve a mutually beneficial scenario.

    This thesis aims to implement and extend upon the \gls{alto} working group, which devised a request-response protocol where authoritative and trustworthy entities provide guidance to applications in the form of network status information and administrative preferences, with the intent of achieving layer cooperation during normal application operations as a means to reach better Internet efficiency through the optimization of infrastructural resourcefulness and consequential minimization of its operational costs.

