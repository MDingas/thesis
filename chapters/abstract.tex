\chapter{Abstract}

    With the ever-increasing Internet usage that is following the start of the new decade, the need to optimize this world-scale network of computers becomes a big priority in the technological sphere that has the number of users rising, as are the \gls{qos} demands by applications in domains such as media streaming or virtual reality.

    In the face of rising traffic and stricter application demands, a better understanding of how \glspl{isp} should manage their assets is needed.
    An important concern regards to how applications utilize the underlying network infrastructure over which they reside.
    Most of these applications act with little regard for \gls{isp} preferences, as exemplified by their lack of care in achieving traffic locality during their operation, which is a preferable feature for network administrators, and that could also improve application performance.
    However, even a best-effort attempt by applications to cooperate will hardly succeed if \gls{isp} policies aren't clearly communicated to them.
    A system to bridge layer interests therefore has much potential in helping achieve a mutually beneficial scenario.

    The main focus of this thesis is the \gls{alto} working group, which was formed by the \gls{ietf} to explore standardizations for network information retrieval. 
    This group specified a request-response protocol where authoritative entities provide resources containing network status information and administrative preferences.
    Sharing of infrastructural insight is done with the intent of enabling a cooperative environment, between the network overlay and underlay, during application operations, to obtain better infrastructural resourcefulness and the consequential minimization of the associated operational costs.

    This work aims to give an overview of the historical network tussle between applications and service providers, present the \gls{alto} working group's project as a solution, as well as implement and extend upon their ideas, and finally verify the developed system's efficiency in a simulation when compared to classical alternatives.

    \bigskip

    \textbf{Keywords:} Application-Layer Traffic Optimization, Content Distribution Networks, Network Optimization, Peer-to-Peer, Traffic Engineering

