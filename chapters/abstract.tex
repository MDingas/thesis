\chapter{Abstract}

    With the ever-increasing global Internet usage that is following the start of the new decade, the need to optimize this world-scale network of computers becomes a big priority in the technological sphere that has the number of users increasing, as are the Quality of Service (QoS) demands by applications in domains such as media streaming or virtual reality.

    In the face of rising traffic and stricter application demands, a better understanding of how Internet Service Providers (ISPs) should maintain their assets is needed.
    As an effort to optimize the Internet, one important concern is how applications utilize the underlying network infrastructure to achieve their goals.
    Peer-to-peer (P2P) applications, for example, have been classically unfavored by ISPs due to their unpredictable and uncontrollable nature, as well as a less than efficient utilization of network resources, resulting in higher costs and non-achievable standards of QoS for ISPs.
    An evident issue is that most of these applications act with little regard for ISP preferences, as can be evidenced by the lack of care by P2P applications in achieving network proximity among neighboring peers, a feature that would be preferable by network administrators and that could also improve application performance.
    Even a best effort attempt by applications to compromise with ISP policies is hard to be successful if such policies aren't clearly communicated to them, and as such a framework to bridge layer interests has much potential.

    This thesis aims to implement and extend upon the Application-Layer Traffic Optimization (ALTO) working group, which devised a request-response protocol where authoritative and trustworthy entities provide guidance to applications in the form of network status information and administrative preferences, with the intent of achieving layer cooperation during normal application operations as a means to achieve better Internet harmony by maximizing optimal resource usage and minimizing costs.

